\chapter{シミュレーション結果}
この章では、典型的な数値シミュレーション結果を示し,板や継手内部をどのように超音波が伝播するかを示す。
はじめに、もっとも基本的なケースである、板内部や表面の波動伝播挙動を示し、モード変換や散乱波の発生状況
について説明する。次に、T字継手を対象とした解析結果を、溶接ビードの形状を考慮しない場合とした場合の
二通りについて見る。これにより、溶接ビードが存在して継手部の形状が若干異なることで、波動伝播にどのような
違いが現れるかを調べる。これらの結果が超音波探傷試験の観点からどのような意味を持つかについては、
次の章で議論する。
\section{平板における波動伝播挙動}
解析モデルを図\ref{fig:}に示す.
このモデルは12mmの厚さの十分に大きな板を想定したもので、その70mmの範囲を解析対象としている。
板両端部の打ち切り位置外側にはPML(perfectly matched layer)吸収領域を設け、無反射条件を課す.
板の上下面は、送信位置以外では、トラクションゼロの境界条件を与えた.
入射波は、平板上面に鉛直力を幅1mmの範囲(4.5$\leq x \leq $5.5)mmに加えて励起した。
その際、鉛直力の時間変化はガウス分布で振幅変調した余弦波で与え、周波数は5MHzとした.
き裂は,水平方向の位置は$x=-12mm$とし,長さ4mm,角度は水平方向から15度とした.
\subsection{入射波の挙動}
図\ref{fig:fig3_1}に,き裂が存在しない場合に生じる波動場、すなわち入射場の様子を示す.
この図は、速度場$\fat{v}(\fat{x},t)$の5つの時刻におけるスナップショットを示し,横軸は
$x$、縦軸$y$とし、各点での粒子速度$\fat{v}(\fat{x},t$の絶対値をカラーマップとして表示
したものである。各々の図には、送信時からの経過時刻が示してある。また、
これらのスナップショットにあらわれている波面の内、明瞭なものについては、モードを
アルファベットで示しており、Pは縦波を、Sは横波、Hはヘッド波,Rは表面波を表している。
(a)にあるように、鉛直力を表面に加えたことで、P波が鉛直方向に強い振幅を持って励起されて
最も早く進展し、その後を、横波S波が続いている。
これらの波面形状は同心円状だが、指向性は異なり、横波はおよそ45度の方向に強く、
鉛直方向で非常に弱い。また、縦波と横波のは波面を結ぶように直線的な波面が認められ、
このような波はヘッド波といわれている。これらの波が進展すると、(b)の図にあるように
板表面近傍に強い表面波(レイリー波)が発生し、横波と分離して現れるようになる。
これら4種類の波は、半無限領域表面に鉛直力を加えたときに発生する波と同じもので、
Lambの解によって理論的に存在と挙動が説明されている。
P波は、(b)に示した時刻では板の下面に達して反射している。
反射の際、P波はP波とS波の両方を発生させる。
この図ではP波の反射によって発生したP波をP-P, P波の反射じにモード変換してS波に転じたものをP-S
として示している。P波は鉛直入射したもの以外は、P、S両方のモードの波を発生させ、
P-Pの後を、伝播速度の遅いP-S波が追いかける形になっている。
(c)では、S波が下面に達した直後の状況を示している。S波についても、P、S両方の反射波
が発生している。この様子は更に時間が経過した(d)の図でより明確であるため、(d)には
S-PとS-Sの波面の所在を示している。なお、(c)ではP波、はやくも2回目の反射を板上面で
起こしており、あまり振幅を低下させることなく伝播していることが分かる。
このように、P波はおよそS波の2倍程度の位相速度で進行するため、S波を追い越しながら
伝播する.(e)にもあるように、板内部では、P波、S波とも、モード変換を伴いつつ、
多重反射を起こすため、板のように単純な形をした部材内部でも、かなり複雑な波動場を
形成する。なお、このような多順反射波が十分な回数繰り返されて干渉を起こす結果が
ガイド波である。ここでの計算条件では、P波の波長が約1.2mm、横波波長が約0.6mm
で、板厚に比べて小さいため、ガイド波が形成されるまでには非常に長い時間が必要とされる
ために、超音波探傷で問題となる観測時間の範囲においては、ガイド波としての解析や解釈
はほとんど意味をなさない。
%--------------------
\begin{figure}[h]
	\begin{center}
	\includegraphics[width=0.8\linewidth]{Figs/plate_inc.eps} 
	\end{center}
	\caption{
		速度場$|\fat{v}(\fat{x},t)|$のスナップショット.
		P,Sは縦波,横波を,Rは表面波,Hはヘッド波を表す.
		ハイフンで繋がれた文字は反射前後でのモードを表す. き裂が存在しない場合.
	} 
	\label{fig:fig3_1}
\end{figure}
\begin{figure}[h]
	\begin{center}
	\includegraphics[width=0.8\linewidth]{Figs/plate_tot.eps} 
	\end{center}
	\caption{
		速度場$|\fat{v}(\fat{x},t)|$のスナップショット.
		P,Sは縦波,横波を,Rは表面波,Hはヘッド波を表す.
		き裂がある場合.
	} 
	\label{fig:fig3_2}
\end{figure}
\begin{figure}[h]
	\begin{center}
	\includegraphics[width=0.8\linewidth]{Figs/plate_sct.eps} 
	\end{center}
	\caption{
		速度場$|\fat{v}(\fat{x},t)|$のスナップショット.
		P,Sは縦波,横波を,Rは表面波,Hはヘッド波を表す.
		き裂がある場合の散乱波成分のみを表示.
	} 
	\label{fig:fig3_3}
\end{figure}
\begin{figure}[h]
	\begin{center}
	\includegraphics[width=0.8\linewidth]{Figs/plate_bwvs.eps} 
	\end{center}
	\caption{
		$x=0\sim35, y=12$mmの位置で得られた波形の走時プロット.
		(a)き裂なし,(b)き裂ありの場合.
		(c)はき裂ありと無しの場合の差分をとって計算した散乱波成分.
	} 
	\label{fig:fig3_4}
\end{figure}
%--------------------
\section{T字継手における波動伝播挙動(余盛り形状を考慮しない場合)}
%--------------------
\begin{figure}[h]
	\begin{center}
	\includegraphics[width=1.0\linewidth]{Figs/T_inc.eps} 
	\end{center}
	\caption{
		速度場$|\fat{v}(\fat{x},t)|$のスナップショット.
		P,Sは縦波,横波を,Rは表面波,Hはヘッド波を表す.
		ハイフンで繋がれた文字は反射前後でのモードを表す. き裂が存在しない場合.
	} 
	\label{fig:fig3_5}
\end{figure}
\begin{figure}[h]
	\begin{center}
	\includegraphics[width=1.0\linewidth]{Figs/T_tot.eps} 
	\end{center}
	\caption{
		速度場$|\fat{v}(\fat{x},t)|$のスナップショット.
		P,Sは縦波,横波を,Rは表面波,Hはヘッド波を表す.
		き裂がある場合.
	} 
	\label{fig:fig3_6}
\end{figure}
\begin{figure}[h]
	\begin{center}
	\includegraphics[width=1.0\linewidth]{Figs/T_sct.eps} 
	\end{center}
	\caption{
		速度場$|\fat{v}(\fat{x},t)|$のスナップショット.
		P,Sは縦波,横波を,Rは表面波,Hはヘッド波を表す.
		き裂がある場合の散乱波成分のみを表示.
	} 
	\label{fig:fig3_7}
\end{figure}
\begin{figure}[h]
	\begin{center}
	\includegraphics[width=1.0\linewidth]{Figs/T_bwvs.eps} 
	\end{center}
	\caption{
		$x=0\sim35, y=12$mmの位置で得られた波形の走時プロット.
		(a)き裂なし,(b)き裂ありの場合.
		(c)はき裂ありと無しの場合の差分をとって計算した散乱波成分.
	} 
	\label{fig:fig3_8}
\end{figure}
%--------------------


