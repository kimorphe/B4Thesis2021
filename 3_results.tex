%\chapter{シミュレーション結果}
この章では、典型的な数値シミュレーション結果を用い,板や継手内部をどのように
超音波が伝播するかを明らかにする。はじめに、最も基本的なケースとして板内部と表面で
の波動伝播挙動を示し、モード変換や散乱波の発生状況を調べる。
次に、T字継手モデルに対して得られた解析結果を示し,継手部で部材が領域が分岐する
ことの効果がどのように現れるかを調べる。
これらの結果が超音波探傷試験の観点からどのような意味を持つかについては、
次章で、より複雑な形状のモデルを用いたシミュレーションの結果を踏まえて議論する。
\section{平板内の波動伝播}
解析モデルを図\ref{fig:fig3_00}に示す.
このモデルは表面き裂を持つ,十分に大きな板を想定したもので、
その70mmの範囲を解析対象としている。板両端部の打ち切り位置外側には
PML(perfectly matched layer)吸収領域を設け無反射条件を課す.
板上下面では、超音波の送信位置以外では応力フリー(トラクション$\fat{t}=\fat{0})$の
境界条件を与えた.入射波は平板上面の幅1mmの範囲(4.5$\leq x \leq $5.5)mmに加えた
一様な鉛直力て励起した。鉛直力の時間変化はガウス分布で振幅変調した余弦波で与え、
その周波数は5MHzとした.き裂の水平位置は$x=-12mm$とし,長さ4mmの鉛直き裂とした.
%--------------------
\begin{figure}[h]
	\begin{center}
	\includegraphics[width=0.7\linewidth]{Figs/model_plate.eps} 
	\end{center}
	\caption{
		鉛直表面き裂を有する平板モデル
	} 
	\label{fig:fig3_00}
\end{figure}
%--------------------
\subsection{入射波の挙動}
図\ref{fig:fig3_1}に,き裂が存在しない場合に生じる波動場、すなわち、入射場の様子を示す.
この図は、速度場$\fat{v}(\fat{x},t)$の5つの時刻におけるスナップショットを示したものである.
横軸は$x$を、縦軸は$y$座標を表し,各点での粒子速度$\fat{v}(\fat{x},t$の絶対値$|\fat{v}|$を
カラーマップとして表示したものである。各々の図には、送信時からの経過時刻が示してある。
これらのスナップショットにある波面のうち明瞭なものについては、伝播モードを
アルファベットで示している.これらの文字は、
Pが縦波を、Sが横波、Hがヘッド波,Rが表面波を表す.
(a)にあるように、鉛直力を表面に加えたことで、P波が鉛直方向に強い振幅を持って励起されて
最も早く進展し、その後ろにS波が続いている。
これらの波面形状はP波、S波とも同心円状だが指向性(放射パターン)は互いに異なる.
S波はおよそ45度の方向に強く、鉛直方向では弱い.逆にP波は、
外力の向きである鉛直下向きには強く放射されているが、水平方向に向かうについれて弱くなる.
P波とS波の波面を結ぶ直線的な波面は,ヘッド波呼ばれるモードである.
これらの波が進展すると、(b)にあるように板表面近傍に発生する表面波(レイリー波)が、
S波と分離して見られるようになる。これら4種類の波は、半無限領域表面に鉛直力を加えたときに
発生する波と同じもので、Lambの問題の解として理論的に存在と挙動が説明されている。
P波は、(b)に示した時刻では板の下面に達して反射し,P波とS波の両方を発生させる。
図ではこれらの反射で生じたP波とS波を,それぞれ,P-P,P-Sとして示している。
P波は鉛直入射した場合を除き,P,S両方のモードを発生させ,
P-Pの後を伝播速度の遅いP-S波が追いかける形になっている。
(c)では、S波が下面に達した直後の状況を示している。S波についても、P、S両方の反射波
(S-PとP-S)が発生している。この様子は更に時間が経過した(d)の図でより明確であるため、
(d)ににみ,S-P波とS-S波の波面位置を示している。
なお、(c)ではP波が2回目の反射を板上面で起こし、振幅をあまり低下させることなく伝播している。
P波はS波のおよそ2倍の位相速度で進行するため、常にS波を追い越しながら伝播する.
(e)にもあるように、板内部では、P波、S波とも、モード変換を伴いつつ多重反射を起こす.
その結果、領域は単純な形をしているにも関わらず,板内部の波動場は
時間経過についれて非常に複雑化つ乱雑なものに見えてくる.
なお、このような多順反射波が十分な回数繰り返されて干渉を起こした結果がガイド波である。
ここでの計算条件では、P波の波長が約1.2mm、横波波長が約0.6mmと厚に比べて小さい.
そのため、ガイド波が形成されるまでには非常に長い時間が必要とされる.このことから、
超音波探傷で問題となる観測時間範囲においては、ガイド波としての解析や解釈
は有効でない.
%--------------------
\begin{figure}[h]
	\begin{center}
	\includegraphics[width=0.8\linewidth]{Figs/plate_inc.eps} 
	\end{center}
	\caption{
		速度場$|\fat{v}(\fat{x},t)|$のスナップショット.
		P,Sは縦波,横波を,Rは表面波,Hはヘッド波を表す.
		ハイフンで繋がれた文字は反射前後でのモードを表す. き裂が存在しない場合.
	} 
	\label{fig:fig3_1}
\end{figure}
%--------------------
\subsubsection{き裂を含む板内部の波動場}
次に,$x=-12$mmに深さ4mmの鉛直き裂が存在する場合の波動場の様子を図\ref{fig:fig3_2}に示す.
この図は図\ref{fig:fig3_1}と同様に、速度場$|\fat{v}|$のスナップショットを示したもので、
き裂が存在すること以外の計算条件は,前の図のものと同じである。
ここでは、き裂位置は白の実線で示してあり、き裂によって生じる散乱波の挙動が示されている.
鉛直力の付加によって励起された波動場は、き裂に到達するまでは図\ref{fig:fig3_1}と全く同じ
挙動を示す。従って(a)の結果は、図\ref{fig:fig3_1}(b)と同じである。図\ref{fig:fig3_2}(a)
では,最初のP波がき裂に到達する直前の状態を示している。この後、P波の一部はき裂で
散乱され、右方向に散乱波が発生する。ただし、散乱波は入射波に比べて小さく、
同じスケールで描画した図\ref{fig:fig3_2}では、いつ励起されてどのように伝播するかがわかりにくい.
このことから、図\ref{fig:fig3_2}では、き裂に到達する主要な波が、各々の時刻に
どの位置にあるかまでを示している。
例えば、下面で一回反射した後のP波(P-P波)は、(b)の時刻にはき裂下側から近づいており、
その下にモード変換で生じたP-S波が続いている。P-S波は(d)の図では、き裂先端部に達しており、その直後
にレイリー波(R波)が、き裂のコーナー部分(開口部)に到達していることが(e)の図に示されている.
\begin{figure}[h]
	\begin{center}
	\includegraphics[width=0.8\linewidth]{Figs/plate_tot.eps} 
	\end{center}
	\caption{
		速度場$|\fat{v}(\fat{x},t)|$のスナップショット.
		P,Sは縦波,横波を,Rは表面波,Hはヘッド波を表す.
		き裂がある場合.
	} 
	\label{fig:fig3_2}
\end{figure}
\subsubsection{散乱波の発生と伝播}
散乱波$\fat{v}^{sc}$は、き裂が存在する場合に生じる波動場$\fat{v}$と、
き裂が存在しない場合の波動場$\fat{v}^{in}$の差として、
\begin{equation}
	\fat{v}^{sc}(\fat{x},t)=\fat{v}(\fat{x},t)-\fat{v}^{in}(\fat{x},t)
	\label{eqn:def_vsc}
\end{equation}
で定義される。
この定義より、き裂がある場合の波動場$\fat{v}$は、
$\fat{v}^{in}+\fat{v}^{sc}$と、入射場と散乱場の和で表されることから、
$\fat{v}$は全波動場、あるいは、より具体的に全速度と呼ばれることがある。
これに対して、$\fat{v}^{in},\fat{v}^{sc}$は、入射波成分、散乱波成分などと呼ばれる。
このような呼び方をするき、図\ref{fig:fig3_1}に示された結果は
全て入射波成分とみなす。従って、入射波には、板表面で反射やモード変換した
P波やS波も含まれる。
実験では、$\fat{v}$と$\fat{v}^{in}$をそれぞれ
求めることは難しいが、数値シミュレーションでは既に示したように
き裂の有無のみを変えたモデルで同じ計算を行うことによって、
$\fat{v}$と$\fat{v}^{in}$を別々に求められる.
そこで、式(\ref{eqn:def_vsc}に従って散乱波$\fat{v}^{sc}$を求め、
その結果をスナップショットとして示したのが図\ref{fig:fig3_3}である。
この図は、図\ref{fig:fig3_2}と同じ5つの時刻における散乱場の様子を
示したものである。ただし、散乱波は入射波に比べて弱いため、
カラー表示する際のスケールは変えてある。
図\ref{fig:fig3_3}(a)の時点では、入射波はき裂に到達していないため
散乱波は生じていない。(b)は、最初のP波が到達した直後で、
3種類のモードの散乱波が現れている.
このうちP1は右方向に伝わる縦波散乱波で、入射波の到来方向に戻ることから
後方散乱波と呼ばれる。
同様に、き裂を超えて左方向に進む散乱波前方散乱波と呼ばれる。
前方散乱波は、本研究の問題設定では観測できないものであることから
以下では議論の対象としない。
S1は、横波の散乱波で、特徴的な円弧状の波面を形成している。
これは、き裂先端で発生した回折波であることから、端部エコーと呼ばれる。
これら縦波後方散乱波と横波端部エコーを結ぶように、わずかにヘッドウェーブ(H1)が現れている。
端部エコーは、き裂先端位置の情報を持つことから、
超音波探傷試験では重要な意味をもつ。

図\ref{fig:fig3_3}(c)は、P-P波によって発生した散乱波(P2,H2)が見られる。
P2とH2は、右下がりのき裂の長さと同程度の波面を示している。
これは、右下方向から入射したP-P波が、き裂端部で回折して発生した
散乱波で、丁度板の上面に到達した時刻にあたる。
これらP2,H2はき裂開口部のコーナでS波に展示、(d)にあるように、板表面方へS2とS3'が、
き裂表面にそって下向きにS3が伝播している。
その後S3は、き裂から離れて伝播を続けることが(e)に示さている。
なお、(e)の時刻は、強い表面波(R)がき裂面に到達した時刻で、この後、顕著な後方散乱波が生じる。
その挙動については次の章であらためて議論する。
%
\subsubsection{走時波形}
超音波探傷試験では、試験体表面の振動が計測可能で、内部の波動場を直接観測
することはできない。そこで、
\begin{equation}
	0\leq x \leq 45 , \ \ y=12[{\rm mm}]
	\label{eqn:Rapt}
\end{equation}
の観測領域で得られた波形がのようであるかを検討する。粒子速度$\fat{v}=(v_1, v_2)$は
ベクトル量で,実験では水平成分$v_1$と鉛直成分$v_2$いずれを測るセンサーも開発されている。
ただし、試験片表面方向の成分である$v_1$は、通常、鉛直成分$v_2$の計測よりも難しく、
圧電センサーの多くは厚み振動のモードを利用して鉛直動を得るものである。
そこで、鉛直成分$v_2$だけが取得できるものとしてここでの議論を進める。
図\ref{fig:fig3_4}は、式(\ref{eqn:Rapt}の観測領域で得られた鉛直方向速度$v_2(\fat{x},t)$
の走時波形である。このプロットは横軸に時間$t$、縦軸に位置$x$をとり、
速度成分$v_2$をカラー表示したものである.超音波探傷分野では、このような波形表示方法を
Bスコープ、あるいはBスキャン波形と呼ぶ.
図\ref{fig:fig3_4}の(a)は、き裂の無い平板で計算した$v_2$、すなわちの入射波成分を、
(b)はき裂が存在する場合の全速度成分を示す。一方、(c)は(b)と(a)の差分として
得られる散乱波成分である.超音波探傷において密に多点計測を行った結果として得られるは、
(b)の全波動場で、 このようなデータからき裂の有無や位置を推定することが求められる。
走時波形に現れる右上がりの曲線は、位置に対して到達時間が遅れることを示し、
右($x>0$)方向への進行波であることを意味する。
図\ref{fig:fig3_4}にはほとんど含まれないが、左上がりの曲線はこの反対で、
左方向$x<0$への進行波を表す.
図\ref{fig:fig3_4}(a)では、30$\mu$sを過ぎることからは、ほとんど鉛直に近い
直線が繰り返す。これは、縦波がどの位置でもほぼ板厚方向へ繰り返し往復して、
定在波に近づくことを示している.
一方、10$\mu$s程度までの早い時間帯では、鉛直方向から傾いた曲線や直線が
現れる。一番はやく現れるのは、板表面に沿って伝播するP波で、この後続く、
非常に大きな振幅を持った直線(R)は表面波である.
これらの波は、板表面に沿って伝わるため、直線の勾配が位相速度を与える。
P波とR波の後には、P-P波が続くが、これ以後のP-S波は、S-P波とも重なる箇所が
あり、走時波形だけから経路を特定することは難しくなる.
き裂がある場合は、(a)に散乱波成分が加わったものとなるため、
基本的なパターンは似たものとなる。しかながら、(c)に示す散乱波成分の中
には非常に強いものもあり、例えばR-Rと示した直線は
表面波によって励起された後方散乱波が、表面波として伝わった結果として現れる
ものである。このような後方散乱波は、この後示す他のモデルでの解析でも
大きな振幅を持った散乱波として普遍的に現れる.
なお、図\ref{fig:fig3_4}(c)のような散乱波成分のみの走時データは、実験では得られない
ことに注意が必要である。しかしながら、数値シミュレーションによって散乱波の挙動を調べて
置けば、実験において目的とする散乱波のモードに合わせて計測位置や計測時間範囲を設定できる。
また、図\ref{fig:fig3_1}-\ref{fig:fig:3_3}のような 結果を参照すれば、走時曲線に現れる
散乱波がどのような入射波モードで励起され、どのような経路を経て到来したかも理解
できる場合が多い。入射波と対応する散乱波の伝播経路が特定できればき裂位置の推定が可能で、
き裂端部の位置を調べることは非破壊検査の観点から特に重要である。
\begin{figure}[h]
	\begin{center}
	\includegraphics[width=0.8\linewidth]{Figs/plate_sct.eps} 
	\end{center}
	\caption{
		速度場$|\fat{v}(\fat{x},t)|$のスナップショット.
		P,Sは縦波,横波を,Rは表面波,Hはヘッド波を表す.
		き裂がある場合の散乱波成分のみを表示.
	} 
	\label{fig:fig3_3}
\end{figure}
\begin{figure}[h]
	\begin{center}
	\includegraphics[width=0.8\linewidth]{Figs/plate_bwvs.eps} 
	\end{center}
	\caption{
		$x=0\sim35, y=12$mmの位置で得られた波形の走時プロット.
		(a)き裂なし,(b)き裂ありの場合.
		(c)はき裂ありと無しの場合の差分をとって計算した散乱波成分.
	} 
	\label{fig:fig3_4}
\end{figure}
%--------------------
\section{T継手における波動伝播挙動(余盛り形状を考慮しない場合)}
%--------------------
図\ref{fig:fig3_01}にT継手部の解析モデルを示す。
フランジは板厚12mm、ウェブの板厚は8mmとし、両者が直角に接合されたT継手の
モデルを考える。領域が分岐することの効果を調べるため、継手ディテールは単純化し、
余盛りやルートギャップなどの形状は考慮しない。き裂と入射波の送信方法は
比較ができるよう前節の平板と同じにする。すなわち、き裂水平位置$x=-12$mm, 
深さ4mm、入射波は$x=5$mmの幅1mmの範囲に加えた等分布荷重で励起する。
荷重の時間変化も前節と同じように設定し,5MHZの波を入射する。
%%
\begin{figure}[h]
	\begin{center}
	\includegraphics[width=0.7\linewidth]{Figs/model_T.eps} 
	\end{center}
	\caption{
		単純化したT継手モデル
	} 
	\label{fig:fig3_01}
\end{figure}
%--------------------
\subsection{入射波の挙動}
はじめに、き裂が無い場合の計算を予め行い入射場$\fat{v}^{in}$を求めた。
この結果の一部を、速度場$\left| \fat{v}^{in}\right|$のスナップショットとして
図\ref{fig:fig3_5}に示す。
この図の(a)は、送信後、最初のP波が継手に達した瞬間を示している。
この時間までは平板と同じであるため、P,S,H波が現れている。
この後、(b)の図にP波、H波は継手右側のコーナで回折波を発生させ、ウェブ側にも
波動場が進展するが、フランジ側の伝播挙動は板の場合とあまり違いがない。
次に、(c)から(d)の時間では、継手に達した表面波とS波がコーナー部分で回折を
起こしながら継手内部に親友している。表面波は部材内部では存在できないため、
表面波はS波となって継手内を進み、(e)の時刻で丁度左側のコーナーに到達する。
なお、フランジの底面で反射したP-P波は、フランジ板厚と縦波位相速度の兼ね合いから、
S波とほとんど同じときに継手右コーナーに達し、その後ウェブ側に入る(図\ref{fig:fig3_5}(d))。
この後、(e)と(f)に見られるように、P-P波はウェブ表面で反射し、新たにP-P-P波、P-P-S波
を励起している。なお、継手内部ではR-S波となった成分は、継手部分を通過後は
再度R波となって進んで行く。(f)の図にはこの様子が僅かながら現れている。
\begin{figure}[h]
	\begin{center}
	\includegraphics[width=1.0\linewidth]{Figs/T_inc.eps} 
	\end{center}
	\caption{
		速度場$|\fat{v}(\fat{x},t)|$のスナップショット.
		P,Sは縦波,横波を,Rは表面波,Hはヘッド波を表す.
		ハイフンで繋がれた文字は反射前後でのモードを表す. き裂が存在しない場合.
	} 
	\label{fig:fig3_5}
\end{figure}
\subsubsection{き裂を含む場合の波動場}
T継手の脚部にき裂がある場合のシミュレーション結果を図\ref{fig:fig3_6}に示す。
この図は、全速度成分$\fat{v}$を示したもので、散乱波成分が含まれる。
散乱波成分は既に述べたように、入射波に比べて非常に小さく同じスケールでは
分布や進展挙動が見えにくい。そこで、図\ref{fig:fig3_6}では、代表的な
モードの波が、散乱波を生ずる前後にどのような状態にあるかを指摘する。
図\ref{fig:fig3_6}(a)は、最初のP波がき裂で散乱された直後の状態を示す。
このときには、他の入射波成分はき裂がない状態と全く同じである。
(b)はこの後、P-P波がき裂に対して下から伝播し、最初にき裂端部に到達する
瞬間の様子である。P-P波はき裂面で反射を起こしつつ(c)の時間にはウェブ内部に進んでいる。
この次の時刻(d)では、R-S波とP-S波がほぼ同時にき裂へ達しつつある。
(e)の図では、R-S波はき裂によってほぼ完全に進路を遮断され、後方散乱波に転じている。
P-S波はき裂端部に到達した後は、(e)と(f)の図にあるようにウェブ側へ進む。
平板の場合との違いは、当然ながらP-P波やP-S波といった反射はがウェブ内に進行する
点で、板上限面での多重反射波が繰返しき裂で散乱されることが無い点にある。
\begin{figure}[h]
	\begin{center}
	\includegraphics[width=1.0\linewidth]{Figs/T_tot.eps} 
	\end{center}
	\caption{
		速度場$|\fat{v}(\fat{x},t)|$のスナップショット.
		P,Sは縦波,横波を,Rは表面波,Hはヘッド波を表す.
		き裂がある場合.
	} 
	\label{fig:fig3_6}
\end{figure}
%%
\subsubsection{散乱波の発生と伝播}
全速度$\fat{v}$から入射波成分$\fat{v}^{in}$を差し引いて求めた、散乱場$\fat{v}^{sc}$の
進展挙動を図\ref{fig:fig3_7}に示す。図\ref{fig:fig3_7}は図\ref{fig:fig3_6}に示した
結果と同じ時刻の散乱波成分の分布を示したものである。ただし、散乱波成分を見えやすく
するため、カラーマップのスケールは互いに異なっている。(a)と(b)の時刻では、
最初にき裂へ直接到達するP波で生じた散乱波P1とS1が見られる。
P1は後方散乱波となって入射方向に戻るものが強く、S1波はほぼ全円状の波面となって
おり、き裂端部エコーとして発生したことが分かる。
この後の時刻(c)では下からのP−P波が縦波P2と、横波散乱波S2を発生させている。
P2、S2とも、き裂面での反射とき裂端部での回折波から構成されている様子が(c)の図から分かる。
この後(d)の図でP2はウェブ内へ入るが、(f)の時刻にかけてS2はウェブ脚部のコーナに
向かう。S2はコーナに到達後再び散乱されるが、(e)の時刻に発生したR-Sに起因する強い横波S3
の影響が強く(f)の時点でその状態を確認することは出来ない。
以上のことから、T継手の場合、P波、P-P波、R-S波の順に散乱波を励起するが、
P-P波はき裂での散乱の後、ウェブの方向に進行するため、フランジ内を繰返し往復するような
散乱波は生じ得ないことが分かる。つまり、継手形状は平板よりも複雑だが、
平板の場合のような周期的な散乱波の発生が少なく、散乱波の伝播挙動はむしろ理解しやすい
ものになっている。
\begin{figure}[h]
	\begin{center}
	\includegraphics[width=1.0\linewidth]{Figs/T_sct.eps} 
	\end{center}
	\caption{
		速度場$|\fat{v}(\fat{x},t)|$のスナップショット.
		P,Sは縦波,横波を,Rは表面波,Hはヘッド波を表す.
		き裂がある場合の散乱波成分のみを表示.
	} 
	\label{fig:fig3_7}
\end{figure}
\subsubsection{走時波形}
図\ref{fig:fig3_8}に、フランジ上面で観測した鉛直動$v_2(\fat{x},t)$の走時波形を示す。
この図にある3つのプロットは、上から順に(a)入射波成分、(b)き裂がある場合の全速度、
(c)散乱波成分を示したmのである。入射波には平板の対応する結果である図\ref{fig:fig3_4}(a)と
類似したパターンになっている。一方、散乱波成分(c)についてみると、目立つものは
10$\mu$sec付近から伸びる表面波の伝播を示す直線(S3-R)しかなく、平板の場合よりも
大きな振幅をもって観測点に到達するエコーが少ないことが分かる。
このS3-Rは、図\ref{fig:fig3_7}(f)にある、横波S3が、フランジ表面で表面波に転じて
長距離を減衰することなく進行するものであることは、各点での到達時間から明らかである。
\begin{figure}[h]
	\begin{center}
	\includegraphics[width=0.8\linewidth]{Figs/T_bwvs.eps} 
	\end{center}
	\caption{
		$x=0\sim35, y=12$mmの位置で得られた波形の走時プロット.
		(a)き裂なし,(b)き裂ありの場合.
		(c)はき裂ありと無しの場合の差分をとって計算した散乱波成分.
	} 
	\label{fig:fig3_8}
\end{figure}
%--------------------


