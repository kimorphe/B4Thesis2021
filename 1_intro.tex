\chapter{はじめに}
\section{研究の背景}
%
%	鋼構造、非破壊検査、超音波探傷、きずの検出と評価
%
我が国には、高度成長期い建設された老朽化の進む橋梁が数多く存在する。その中には鋼橋梁も多数含まれる。
鋼橋梁の劣化は主として腐食と疲労によって進行する。
疲労は、繰り返し載荷によって疲労き裂が発生、進展することにより応力集中や断面欠損によって橋梁の耐荷力が低下するものある。
き裂の進展は、鋼材の降伏応力よりも小さな応力レベルでも生じる。
疲労き裂は溶接継手部で発生することから、疲労損傷の予防や補修のためには、
溶接部におけるき裂の発生や進展挙動について理解することが必要とされる。
その際、き裂の起点となるブローホールや融合不良といった溶接欠陥の有無や、
すでにき裂が生じている場合にはき裂自身の位置や大きさを知ることが重要となる。
多くの場合、溶接欠陥や疲労き裂は、溶接継手内部に存在する。
き裂に関しては、部材表面に開口しているような場合(表面き裂)として存在する場合もある。
しかしながら、き裂進展部に当たる先端は部材内部にあり、目視検査だけでは大きさや向きを
完全に知ることはできない。さらに、き裂の開口部が閉断面リブの内側にあるような場合には、部材表面で
あっても直接観察することはできない。
以上のことから、溶接継手部の現場での探傷には固体内部の状態を観察
することのできる各種非破壊検査法が用いられている。
代表的な非破壊検査法には、磁粉探傷法、X線透過試験、超音波探傷法、渦電流法、サーモグラフィー
による欠陥検出法などが挙げられる。
この内、磁粉探傷法は目視検査の一種であり、内部のき裂を検出することはできない。
また、X線透過試験は厚板には用いることができず、放射線遮蔽の問題もあることから
現場探傷には不向きである。一方、渦電流法やサーモグラフィーは、
表面付近のき裂検出に適するものの、内部き裂の検出やサイズの評価は得意でなく、
検査すべき領域が露出していない状況では適用できない。
これに対して超音波探傷試験では、固体内部を伝播する弾性波の一種である
超音波エコーを観察することでき裂の検出や評価を行う。
超音波探傷のための装置は原理的には単純かつ比較的安価に構築することができ、
安全上の問題もない。
また、検査に用いる超音波のモードや周波数を適切に選べば、内部欠陥の検出や評価
が可能で、これらの点において、現場探傷のための非破壊検査法として他の手法に
無い利点がある.\\

%
%	UTの原理と課題(複数経路、形状エコー)
%
超音波探傷試験では、圧電素子やレーザを使って超音波を試験体に励起する。
超音波は、き裂や介在物など、密度や剛性が変化する欠陥(きず)部位で反射、散乱
されて超音波エコーが発生する。きずからの超音波エコーを物体表面で観察することにより、
物体内部に不均一部や界面が存在することを検知することができる。
さらに、超音波の伝播速度が既知であれば、入射波の送信時刻とエコーの到達時刻
から伝播時間が求まり、伝播時間からきずまでの距離を知ることができる。
このような作業を多数の点で行い距離情報を集めれば、超音波エコー波形から、
きずの正確な位置が得られる。また、得られた波形をトモグラフィー処理するなどして
可視化すれば、欠陥の像(イメージ)として分かりやすく表示することもできる。
このように超音波擔傷法の原理は、エコーの伝播時間を距離に換算して位置特定を
行うという単純なものである。
一方で、超音波探傷法を溶接継手の検査に適用する場合、伝播時間や距離の決定は
簡単なものではない。
まず、超音波エコーの発生源は、検出すべききずだけに限らない。
例えば、溶接止端部や部材端の角部など、形状の変化が大きい箇所では回折波が
生じ、計測波形には複数のエコーが同時に現れる。これら複数のエコーのうち
きずからのエコーは通常、微弱で、不要なエコーに隠されて探傷波形上で簡単には
特定できない。さらに、固体内部の超音波の伝播挙動は複雑で、縦波、横波に加え、
表面波や界面波を含め複数のモードが発生する。これらのモードは反射や散乱時に
互いに変換が起きる(モード変換)ため、一つのきずを検出する場合も、
可能なエコー伝播経路は複数存在する。また、それらの複数の経路を経たエコーの
いうれが実際に観測されるかを判定する簡単な基準も存在しない。\\

%
%	エコー振幅や伝播経路の解析(波線理論と数値波動解析)
%
以上のように、超音波による溶接継手の探傷では、不要なエコーときずエコーの
分別と、エコー伝播経路の特定が課題となる。可能なエコー伝播経路は、概ね波線理論(ray theory)
によって調べることができる。波線理論は波動伝播に関する近似理論で、高周波数の波は均質材中を
波線(ray)と呼ばれる直線に沿って伝播し、物体の表面ではスネルの法則によって反射、
屈折、モード変換を起こすと考える。波線理論は伝播時間や経路に関する限り、フェルマーの原理
やホイヘンスの原理とも同じ結果を与える。一方、波線に沿った振幅の変化を計算することには
困難が多く、特に、回折波や表面波が発生する状況では限られたケースでしか波線理論解析が有効でない。
このことは、波線理論によって、可能な経路とその伝播時間の見積りはある程度可能だが、
振幅値を予想してエコー強度を見積り、特定の経路を経たエコーが観測されるかどうかを
判断するのは難しいことを意味する。
これに対して、有限要素法や差分法による数値波動解析では、縦波や横波と、
表面波等の非実体波を区別することなく、任意の物性値や形状で波動場の計算が可能である。
固体中の超音波は弾性波の一種で振幅も小さく、微小変形理論を適用し線形弾性体として扱うことができる。
従って、線形弾性固体の運動方程式を与えられた初期値、境界値の元で計算することで、
任意の時間と位置における超音波の振幅や位相を計算することができる。
こういった波動解析は、計算効率の問題は依然として大きいものの、これまでの
数値解析技術と計算機環境の整備により、今日では難しいものではない。
実際、これまでに数多くの超音波探傷試験の数値シミュレーションが行われている。
一方で、数値波動解析の結果では、種々のモードが混在した複雑な波動場が得られるため、
数値解析結果自体を解釈することも結局は簡単ではない。特に、数値計算で予想されたエコーが
どのような経路を辿ったものかについては、数値解析結果に明示的に現れる訳でなく、
数値波動解析が超音波探傷結果の理解を自動的に深めてくれる状況には至っていない。\\

%
%	溶接継手部の超音波探傷(形状エコ-、送受信位置、モードの問題)
%
きずからのエコ-が、周辺で発生する不要なエコーで隠されて検出困難と状況は、
ごく簡単な継ぎ手形状でも発生する。例えば、きずの無い板内部にどのうような
波動場が生じるかは理論的に詳しく調べられ、薄板ではLamb波と呼ばれるガイド波
が生じることが知られている。突き合わせ溶接継手は、溶接ビードを除き概ね板材と
みなすことができるため、このような状況に当てはまる。一方、基本的な継ぎ手形式
であるT字継手では、きずが存在しない場合も波動場の理論解は得られない。そのため、波
動場の解析は数値解析に依らざるを得ない。T字継手では、継手部分の角部で回折波が生じる。
き裂が存在する場合には、角部からの回折波がき裂でも散乱され、非常に複雑な挙動を示す。
さらに、実際の継手部には溶接ビードが存在するため、ビード内部や表面での反射は、
継手周辺の平板部分とは大きくことなる波動伝播形態を示す。
これらのことが相俟って、継手の形状としては単純なT字継手でも、超音波伝播挙動は極めて
複雑になり得る。従って、継手部周辺での波動伝播挙動を正確に理解し、観測可能なエコーの
発生メカニズムと伝播経路を把握することは超音波探傷試験の適切な実施の上で重要である。
\section{研究の目的}
以上を踏まえ本研究では、T字溶接継手におけるきずエコーの発生および伝播挙動を
理解することを目的とした詳細な数値シミュレーションを行う。
具体的には、T字継手の角部から発生し、フランジ側に伸びるき裂を対象とし、
き裂に到達する入射波と、それに対して発生する散乱波の挙動を調べる。
超音波の送受信はフランジの一方の面で、き裂に対して片側からのみ
可能とし、極めて制約の厳しい計測条件を想定したシミュレーションを行う。
ただし、受信は密に配置された多数の点でアレイ観測できるものとする。
数値シミュレーションにはFDTD法(時間領域有限差分法finit difference time domain method)を
用い、模擬探傷波形データを合成する。
この波形中に含まれる主要なき裂エコーが、どのようにして発生、伝播したかを
FDTD法によって部材内部の波動場の詳細な観察から明らかにする。
これら一連の計算結果を踏まえ、継手とき裂に対して、どのような送受条件で探傷
を行うべきかについて、一般性のある提案を行う。
\section{本論文の構成}
本論文の構成を以下に述べる。
本章で述べた研究背景と目的に続き,第二章では、超音波探傷シミュレーションの問題設定と
数値波動解析法について示す。ここでは、想定する継手形状や探傷条件、
波動場の支配方程式と数値解析法について述べる。
第三章では、代表的なシミュレーション結果を示す。シミュレーションは、
比較のために平板とT字継手で行い、各々、どのような入射波、散乱波が
発生するかを見る。また、この章では、シミュレーション結果の表示と解釈の方法について示す。
第4章では、より現実的なモデルとして、溶接部の余盛形状をモデルに取り込んだ場合の
シミュレーションを行う。これは、近年、疲労損傷が問題となっている、
鋼床版Uリブのすみ肉溶接部に発生するき裂の探傷を模擬したものである。
溶接部の複雑な形状がエコー形成に与える影響を調べ、どのような送受条件で
有用なきずエコーが観測可能であるかをこのシミュレーションを通じて明らかにする。
最終、第5章では本研究で得られた知見をまとめ、結論と今後の課題を示す。
なお、本研究では計算効率の制約のため全ての数値シミュレーションを2次元問題として行う.

