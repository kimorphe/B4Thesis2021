%\chapter{はじめに}
\section{研究の背景}
%
%	鋼構造,非破壊検査,超音波探傷,きずの検出と評価
%
我が国には,高度成長期に建設され老朽化の進む橋梁が数多く存在し,そこには鋼橋梁も多数含まれる. 
鋼橋梁の劣化は主として腐食と疲労で進行する.
疲労は,繰り返し載荷によってき裂が発生,進展する劣化現象で,応力集中や断面欠損で橋梁の耐荷力を低下させる.疲労き裂は溶接継手部で発生することから,疲労損傷の予防や補修のためには,溶接部におけるき裂の発生や進展挙動について理解することが必要となる\cite{Miki}.
その際,き裂の起点となるブローホールや融合不良といった溶接欠陥の有無や,すでにき裂が生じている場合はき裂自身の位置や大きさを知ることが重要となる.
き裂に関して言えば,部材表面に開口した表面き裂として存在する場合もあるが,き裂進展部に当たる先端は部材内にあり,目視検査だけで大きさや向きを知ることはできない.
さらに,き裂の開口部が,例えば閉断面リブ内側にあるような場合には,部材表面であってもき裂位置を直接観察することはできない.
以上のことから,溶接継手部の探傷には固体内部の状態を観察することのできる各種非破壊検査法が用いられる.
代表的な非破壊検査法には,磁粉探傷法,X線透過試験,超音波探傷法,渦電流法,サーモグラフィー
による欠陥検出法などがある.磁粉探傷法は目視検査の一種で,内部き裂を検出することはできない.
また,X線透過試験は厚板には適用できず,放射線遮蔽の問題もあり現場探傷には不向きである.
一方,渦電流法やサーモグラフィーは表面付近のき裂検出に適するものの,内部き裂の検出やサイズの評価は得意でない.これに対して超音波探傷試験では,固体内部を伝播する弾性波の一種である超音波エコーを観察することでき裂の検出や評価を行う.超音波探傷のための装置は原理的には単純かつ比較的安価に構築することができ,
安全上の問題もない.また,検査に用いる超音波のモードや周波数を適切に選べば,内部欠陥の検出や評価が可能で,これらの点で現場探傷のための非破壊検査法として他の手法には無い利点がある\\

%
%	UTの原理と課題(複数経路,形状エコー)
%
超音波探傷試験では,圧電素子やレーザを使って超音波を試験体に励起する.
超音波は,き裂や介在物など,密度や剛性が変化する欠陥(きず)部位で反射,散乱されて超音波エコーが発生する.このことから,きずからの超音波エコーを物体表面で観察することにより,物体内部に不均一部や界面が存在することを検知できる.
さらに,超音波の伝播速度が既知であれば,入射波の送信時刻とエコー到達時刻から伝播時間が求まり,伝播時間からきずまでの距離を知ることができる.このような作業を多数の点で行い距離情報を集めれば,超音波エコー波形から,きずの正確な位置が得られる.また,得られた波形をトモグラフィー処理するなどして可視化すれば,欠陥の像(イメージ)として分かりやすく表示することもできる.このように超音波探傷法は,エコーの伝播時間を距離に換算して位置特定を行うという原理的には単純なものである\cite{US}.一方で,超音波探傷法を溶接継手の検査に適用する場合,伝播時間や距離の決定は必ずしも簡単ではない.
まず,超音波エコーの発生源は検出すべきき裂だけに限られない.
例えば,溶接止端部や部材端の角部など,急な形状変化がある箇所では回折波が生じ,計測波形に複数のエコーが場合いよっては同時刻に現れる.これら複数のエコーのうち,きずからのエコーは通常,微弱で,不要なエコーに隠され探傷波形上、簡単には特定できない.さらに,固体内部の超音波の伝播挙動は複雑で,縦波や横波に加えて,
表面波や界面波も含む複数のモードが発生する\cite{JDA}.これらの波動モードは反射や散乱時に互いに変換が起きる(モード変換)ため,一つのきずを検出する場合も,可能なエコー伝播経路が複数存在する.また,それら複数の経路を経たエコーのいずれが実際に観測されるかを判定する簡単な基準も存在しない.\\

%
%	エコー振幅や伝播経路の解析(波線理論と数値波動解析)
%
以上のように,超音波による溶接継手の探傷では,不要なエコー(形状エコー)ときずエコーの分別と,エコー伝播経路の特定が課題となる.可能なエコー伝播経路は,概ね波線理論(ray theory)によって調べることができる\cite{JDA}.波線理論は波動伝播に関する近似理論で,高周波数の波は均質材中を波線(ray)と呼ばれる直線に沿って伝播すると考える.また,物体表面ではスネルの法則によって反射や屈折,モード変換を起こすと考える.波線理論は伝播時間や経路に関する限り,フェルマーの原理やホイヘンスの原理と同じ結果を与える.一方,波線に沿った振幅変化を計算するのは困難が多く,特に,回折波や表面波が発生する状況では限られたケースでしか波線理論解析が有効でない.このことは,可能な経路とその伝播時間の見積りは波線理論である程度可能だが,振幅値を予想してエコー強度を予想し,特定の経路を経たエコーが観測されるかどうかを波線理論的な検討から判断するのは難しいことを意味する.
これに対して,有限要素法や差分法による数値波動解析では,縦波や横波,表面波等の非実体波を区別することなく,任意の物性値や形状で波動場の計算が可能である.
固体中の超音波は弾性波の一種で振幅も小さく,微小変形理論を適用し線形弾性体として扱うことができる.
従って,線形弾性固体の運動方程式を与えられた初期値,境界値の元で計算することで,任意の時間と位置における超音波の振幅や位相を計算することができる.こういった波動解析は,計算負荷は依然として大きいものの,これまでの数値解析技術と計算機環境の整備により,今日では実施自体はそれほど難しいものではない.しかしながら,数値波動解析の結果では,種々のモードが混在した複雑な波動場が得られるために,数値解析結果を解釈することも結局は簡単でない.特に,数値計算で予想されたエコーがどのような経路を辿ったものかについては,数値解析結果に明示的に示される訳でなく,超音波探傷結果の理解を自動的に深めてくれる状況には至っていない.\\

%
%	溶接継手部の超音波探傷(形状エコ-,送受信位置,モードの問題)
%
きずエコ-が,周辺で発生する形状エコーで隠され検出困難となる状況は,ごく簡単な継ぎ手形状でも発生する.きずの無い板であれば、内部にどのうような波動場が生じるかは理論的に詳しく調べられ,薄板ではLamb波と呼ばれるガイド波が生じることが知られている.突き合わせ溶接継手は,溶接ビードの部分を除けば、概ね板材とみなすことができるため,この状況に当てはまる.一方,基本的な継ぎ手形式であるT継手では,きずが存在しない場合も波動場の理論解は得られない.そのため,波動場解析は数値解析に依らざるを得ない.
T継手では,継手部分の角部で回折波が生じる.さらに,き裂が存在する場合には,角部からの回折波がき裂でも散乱され,複雑な多重散乱場を形成する.また,実際の継手部には溶接ビードが存在するため,ビード内部や表面での反射は,継手周辺の平板部分と大きくことなる波動伝播形態を示す.
これらのことが相俟って,継手形状としては単純なT継手でも,超音波伝播挙動は複雑になり得る.
従って,継手周辺での波動伝播挙動を正確に理解し,観測可能なエコーの発生メカニズムと伝搬挙動を把握することは超音波探傷試験の適切な実施の上で重要である.
\section{研究の目的}
以上を踏まえ本研究では,T溶接継手におけるきずエコーの発生および伝播挙動を理解することを目的とした数値シミュレーションを行う.具体的には,T継手角部から発生してフランジ側へ伸びるき裂を対象として,き裂に到達する入射波と,それによって励起されるする散乱波の挙動を調べる.超音波の送受信はフランジの一方の面で,き裂に対して片側からのみ可能と仮定する、制約の厳しい計測条件を想定したシミュレーションを行う.
数値シミュレーションにはFDTD法(時間領域有限差分法finit difference time domain method)\cite{FDTD1},\cite{FDTD2}を用い,模擬探傷波形データを合成する.この波形中に含まれる主要なき裂エコーが,いつどのようにして発生,伝搬したかをFDTD法によって部材内部の波動場観察から明らかにする.
これら一連の計算結果の示す知見を,T継手の超音波探傷試験の観点から整理,考察する.
\section{本論文の構成}
本論文の構成を以下に述べる.
本章で述べた研究背景と目的に続き,第二章では超音波探傷シミュレーションの問題設定と数値波動解析法について示す.ここでは,想定する継手形状や探傷条件,波動場の支配方程式と数値解析法について述べる.
第三章では,簡単な形状のモデルで行ったシミュレーションの結果を示す.シミュレーションは,
基礎モデルとして平板と簡易なT継手モデルで行い,各々,どのような入射波と散乱波が発生するかを見る.
第4章では,より現実的なモデルとして,T溶接継手の余盛形状をモデルに取り込んだ詳細なシミュレーションを行う.これは,近年,疲労損傷が問題となっている,鋼床版Uリブのすみ肉溶接部に発生するき裂の探傷を模擬したものである\cite{Urib1},\cite{Urib2},\cite{Urib3}.溶接部の複雑な形状がエコー形成に与える影響を調べ,どのような送受条件で有用なきずエコーが観測可能であるかをこのシミュレーションを通じて明らかにする.
最終,第5章では本研究で得られた知見をまとめ,結論と今後の課題を示す.なお,本研究では計算効率の制約のため全ての数値シミュレーションを2次元問題として行う.

