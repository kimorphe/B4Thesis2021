\documentclass[11pt,a4j]{mybook2}
\usepackage[top=2.5cm, bottom=2.5cm, left=2cm, right=2cm]{geometry}
%\usepackage{showkeys}
%\documentclass[11pt,a4j]{jbook}
%\usepackage{graphicx,wrapfig}
\usepackage{graphicx,titlesec}
%\usepackage{tocloft} %目次の調整
%\setlength{\topmargin}{-1.5cm}
%\setlength{\textwidth}{16.5cm}
%\setlength{\textheight}{25.2cm}
\newlength{\minitwocolumn}
\setlength{\minitwocolumn}{0.5\textwidth}
\addtolength{\minitwocolumn}{-\columnsep}
%\addtolength{\baselineskip}{-0.1\baselineskip}
%
\def\Mmaru#1{{\ooalign{\hfil#1\/\hfil\crcr
\raise.167ex\hbox{\mathhexbox 20D}}}}
%
\newcommand{\fat}[1]{\mbox{\boldmath $#1$}}
\newcommand{\D}{\partial}
\newcommand{\w}{\omega}
\newcommand{\ga}{\alpha}
\newcommand{\gb}{\beta}
\newcommand{\gx}{\xi}
\newcommand{\gz}{\zeta}
\newcommand{\vhat}[1]{\hat{\fat{#1}}}
\newcommand{\spc}{\vspace{0.7\baselineskip}}
\newcommand{\halfspc}{\vspace{0.3\baselineskip}}
\bibliographystyle{unsrt}
\newcommand{\twofig}[2]
 {
   \begin{figure}
     \begin{minipage}[t]{\minitwocolumn}
         \begin{center}   #1
         \end{center}
     \end{minipage}
         \hspace{\columnsep}
     \begin{minipage}[t]{\minitwocolumn}
         \begin{center} #2
         \end{center}
     \end{minipage}
   \end{figure}
 }
%\titleformat{\chapter}[display]{\normalfont\normalsize}{\chaptertitlename \thechapter 章}{20pt}{\normalsize}
%{\normalsize}
%\vspace*{\baselineskip}
%\renewcommand{\cfttoctitlefont}{\hfill\normalsize\bfseries}
%\renewcommand{\cftaftertoctitle}{\hfill\null}
\renewcommand{\labelenumi}{(\arabic{enumi})}

\title{
\vspace{20mm}
T字溶接継手部のき裂による\\
超音波エコーの励起・伝搬メカニズム解析
\vspace{5mm}
Generation and Propagation Mechanisms of Ultrasonic Echoes \\
due to a Crack in a T-shaped Welding Joint
\vspace{60mm}
}
%\date{\today}
\date{2022年2月10日}
\author{
	\vspace{40mm}
岡山大学環境理工学部\\
環境デザイン工学科 10430231\\
	永井 瑞希
}

%\makeatletter
%\def\@evenfoot{\hfil -\thepage- \hfil}
%\makeatother
%\makeatletter
%\def\@oddfoot{\hfil -\thepage- \hfil}
%\makeatother
%\makeatletter
%\def\@oddeven{}
%\makeatother

\begin{document}
\maketitle
%-------------------------
\begin{center}
\begin{minipage}{15cm}
\begin{center}
	{\bf 要旨}
\end{center}
本研究では,溶接部の超音波探傷試験の信頼性を担保する上で有用な知見を得ることを目的に,T継手のき裂による超音波エコーの励起・伝播メカニズムを,2次元数値シミュレーションで調べた.シミュレーションには時間領域差分法(FDTD法)を用い,溶接ルートから進展したき裂を模擬した解析モデルで,散乱波の発生と伝播挙動を検討した.
その際,全てのシミュレーションをき裂の有無だけを変えたモデルで行い,両者の差から散乱波成分を抽出した.また,各種散乱波の成因を明らかにするために,入射波がき裂に到達する時間を超音波モード毎に調べ,散乱波の発生タイミングと入射波およびモードの対応を明らかにした.さらに,散乱波による供試体表面の鉛直動を走時波形として示し,超音波探傷試験で観測しうるエコー成分と,き裂から観測点への伝搬経路を明らかにした.以上の結果,大きな振幅を有する表面波と,到達時間が早く複数経路から到達する縦波,それぞれの後方散乱波がき裂検出に有効であることが分かった.さらに,余盛り表面がレンズ効果を果たすことで,強い横波散乱波が発生すること現象を見出し,このエコーを用いることで新たな探傷法開発につながるとの知見を得た.これらの検討は,今後,T継手のき裂探傷条件を最適化する上で有用な情報となる.
	\vspace{15mm}
\begin{center}
	{\bf ABSTRACT}
\end{center}
This study investigats the mechanisms of ultrasonic echo generation and propagation due to 
a crack in a T-shaped welding joint, aiming to gain an insight to enhance the quality of the 
ultrasonic flaw inspection.
For this purpose, 2D finite difference time-domain methods were performed to simulate ultrasonic 
scattering by a vertical crack exteding from the corner of a T-weld joint. 
To separate the scatterd wave component from the total field accurately, all the simulations were 
performed both on the numerical model with and without the crack. The evolution of the incident 
fields were also inviestingatd mode by mode to inventry the major scattered waves in conjuction 
with the incident wave components that excite them. Furthermore, measured ultrasonic signals are 
synthesized as a travel time plot to identify the scattered wave compoents that are potentially 
useful for ultrasonic flaw detection. As a result of the numerical study, strong Rayleigh surface 
wave and the fast traveling P-waves both back-scattered from the crack face were found to 
 emerge as a two most notable ultrasoic echoes. Moreover, it was found for the first time that 
the welding bead can work as an acoustic lenz that focuses the incident waves finely to the crack, 
which  generates the ultrasonic echo of appreciable intensity.
The findings above are useful in optimizing the measurement conditions for the ultrasonic 
inspection of T-weld joint.
\end{minipage}
\end{center}
%-------------------------
\tableofcontents
\frontmatter
\mainmatter
%%%%%%%%%%%%%%%%%%%%%%%%%%%%%%%%%%%%%%%%%%%%%%%%%%%%%%%%%%%%%%%%
\chapter{はじめに}
	%\chapter{はじめに}
\section{研究の背景}
%
%	鋼構造,非破壊検査,超音波探傷,きずの検出と評価
%
我が国には,高度成長期に建設され老朽化の進む橋梁が数多く存在し,そこには鋼橋梁も多数含まれる. 
鋼橋梁の劣化は主として腐食と疲労で進行する.
疲労は,繰り返し載荷によってき裂が発生,進展する劣化現象で,応力集中や断面欠損で橋梁の耐荷力を低下させる.疲労き裂は溶接継手部で発生することから,疲労損傷の予防や補修のためには,溶接部におけるき裂の発生や進展挙動について理解することが必要となる\cite{Miki}.
その際,き裂の起点となるブローホールや融合不良といった溶接欠陥の有無や,すでにき裂が生じている場合はき裂自身の位置や大きさを知ることが重要となる.
き裂に関して言えば,部材表面に開口した表面き裂として存在する場合もあるが,き裂進展部に当たる先端は部材内にあり,目視検査だけで大きさや向きを知ることはできない.
さらに,き裂の開口部が,例えば閉断面リブ内側にあるような場合には,部材表面であってもき裂位置を直接観察することはできない.
以上のことから,溶接継手部の探傷には固体内部の状態を観察することのできる各種非破壊検査法が用いられる.
代表的な非破壊検査法には,磁粉探傷法,X線透過試験,超音波探傷法,渦電流法,サーモグラフィー
による欠陥検出法などがある.磁粉探傷法は目視検査の一種で,内部き裂を検出することはできない.
また,X線透過試験は厚板には適用できず,放射線遮蔽の問題もあり現場探傷には不向きである.
一方,渦電流法やサーモグラフィーは表面付近のき裂検出に適するものの,内部き裂の検出やサイズの評価は得意でない.これに対して超音波探傷試験では,固体内部を伝播する弾性波の一種である超音波エコーを観察することでき裂の検出や評価を行う.超音波探傷のための装置は原理的には単純かつ比較的安価に構築することができ,
安全上の問題もない.また,検査に用いる超音波のモードや周波数を適切に選べば,内部欠陥の検出や評価が可能で,これらの点で現場探傷のための非破壊検査法として他の手法には無い利点がある\\

%
%	UTの原理と課題(複数経路,形状エコー)
%
超音波探傷試験では,圧電素子やレーザを使って超音波を試験体に励起する.
超音波は,き裂や介在物など,密度や剛性が変化する欠陥(きず)部位で反射,散乱されて超音波エコーが発生する.このことから,きずからの超音波エコーを物体表面で観察することにより,物体内部に不均一部や界面が存在することを検知できる.
さらに,超音波の伝播速度が既知であれば,入射波の送信時刻とエコー到達時刻から伝播時間が求まり,伝播時間からきずまでの距離を知ることができる.このような作業を多数の点で行い距離情報を集めれば,超音波エコー波形から,きずの正確な位置が得られる.また,得られた波形をトモグラフィー処理するなどして可視化すれば,欠陥の像(イメージ)として分かりやすく表示することもできる.このように超音波探傷法は,エコーの伝播時間を距離に換算して位置特定を行うという原理的には単純なものである\cite{US}.一方で,超音波探傷法を溶接継手の検査に適用する場合,伝播時間や距離の決定は必ずしも簡単ではない.
まず,超音波エコーの発生源は検出すべきき裂だけに限られない.
例えば,溶接止端部や部材端の角部など,急な形状変化がある箇所では回折波が生じ,計測波形に複数のエコーが場合いよっては同時刻に現れる.これら複数のエコーのうち,きずからのエコーは通常,微弱で,不要なエコーに隠され探傷波形上、簡単には特定できない.さらに,固体内部の超音波の伝播挙動は複雑で,縦波や横波に加えて,
表面波や界面波も含む複数のモードが発生する\cite{JDA}.これらの波動モードは反射や散乱時に互いに変換が起きる(モード変換)ため,一つのきずを検出する場合も,可能なエコー伝播経路が複数存在する.また,それら複数の経路を経たエコーのいずれが実際に観測されるかを判定する簡単な基準も存在しない.\\

%
%	エコー振幅や伝播経路の解析(波線理論と数値波動解析)
%
以上のように,超音波による溶接継手の探傷では,不要なエコー(形状エコー)ときずエコーの分別と,エコー伝播経路の特定が課題となる.可能なエコー伝播経路は,概ね波線理論(ray theory)によって調べることができる\cite{JDA}.波線理論は波動伝播に関する近似理論で,高周波数の波は均質材中を波線(ray)と呼ばれる直線に沿って伝播すると考える.また,物体表面ではスネルの法則によって反射や屈折,モード変換を起こすと考える.波線理論は伝播時間や経路に関する限り,フェルマーの原理やホイヘンスの原理と同じ結果を与える.一方,波線に沿った振幅変化を計算するのは困難が多く,特に,回折波や表面波が発生する状況では限られたケースでしか波線理論解析が有効でない.このことは,可能な経路とその伝播時間の見積りは波線理論である程度可能だが,振幅値を予想してエコー強度を予想し,特定の経路を経たエコーが観測されるかどうかを波線理論的な検討から判断するのは難しいことを意味する.
これに対して,有限要素法や差分法による数値波動解析では,縦波や横波,表面波等の非実体波を区別することなく,任意の物性値や形状で波動場の計算が可能である.
固体中の超音波は弾性波の一種で振幅も小さく,微小変形理論を適用し線形弾性体として扱うことができる.
従って,線形弾性固体の運動方程式を与えられた初期値,境界値の元で計算することで,任意の時間と位置における超音波の振幅や位相を計算することができる.こういった波動解析は,計算負荷は依然として大きいものの,これまでの数値解析技術と計算機環境の整備により,今日では実施自体はそれほど難しいものではない.しかしながら,数値波動解析の結果では,種々のモードが混在した複雑な波動場が得られるために,数値解析結果を解釈することも結局は簡単でない.特に,数値計算で予想されたエコーがどのような経路を辿ったものかについては,数値解析結果に明示的に示される訳でなく,超音波探傷結果の理解を自動的に深めてくれる状況には至っていない.\\

%
%	溶接継手部の超音波探傷(形状エコ-,送受信位置,モードの問題)
%
きずエコ-が,周辺で発生する形状エコーで隠され検出困難となる状況は,ごく簡単な継ぎ手形状でも発生する.きずの無い板であれば、内部にどのうような波動場が生じるかは理論的に詳しく調べられ,薄板ではLamb波と呼ばれるガイド波が生じることが知られている.突き合わせ溶接継手は,溶接ビードの部分を除けば、概ね板材とみなすことができるため,この状況に当てはまる.一方,基本的な継ぎ手形式であるT継手では,きずが存在しない場合も波動場の理論解は得られない.そのため,波動場解析は数値解析に依らざるを得ない.
T継手では,継手部分の角部で回折波が生じる.さらに,き裂が存在する場合には,角部からの回折波がき裂でも散乱され,複雑な多重散乱場を形成する.また,実際の継手部には溶接ビードが存在するため,ビード内部や表面での反射は,継手周辺の平板部分と大きくことなる波動伝播形態を示す.
これらのことが相俟って,継手形状としては単純なT継手でも,超音波伝播挙動は複雑になり得る.
従って,継手周辺での波動伝播挙動を正確に理解し,観測可能なエコーの発生メカニズムと伝搬挙動を把握することは超音波探傷試験の適切な実施の上で重要である.
\section{研究の目的}
以上を踏まえ本研究では,T溶接継手におけるきずエコーの発生および伝播挙動を理解することを目的とした数値シミュレーションを行う.具体的には,T継手角部から発生してフランジ側へ伸びるき裂を対象として,き裂に到達する入射波と,それによって励起されるする散乱波の挙動を調べる.超音波の送受信はフランジの一方の面で,き裂に対して片側からのみ可能と仮定する、制約の厳しい計測条件を想定したシミュレーションを行う.
数値シミュレーションにはFDTD法(時間領域有限差分法finit difference time domain method)\cite{FDTD1},\cite{FDTD2}を用い,模擬探傷波形データを合成する.この波形中に含まれる主要なき裂エコーが,いつどのようにして発生,伝搬したかをFDTD法によって部材内部の波動場観察から明らかにする.
これら一連の計算結果の示す知見を,T継手の超音波探傷試験の観点から整理,考察する.
\section{本論文の構成}
本論文の構成を以下に述べる.
本章で述べた研究背景と目的に続き,第二章では超音波探傷シミュレーションの問題設定と数値波動解析法について示す.ここでは,想定する継手形状や探傷条件,波動場の支配方程式と数値解析法について述べる.
第三章では,簡単な形状のモデルで行ったシミュレーションの結果を示す.シミュレーションは,
基礎モデルとして平板と簡易なT継手モデルで行い,各々,どのような入射波と散乱波が発生するかを見る.
第4章では,より現実的なモデルとして,T溶接継手の余盛形状をモデルに取り込んだ詳細なシミュレーションを行う.これは,近年,疲労損傷が問題となっている,鋼床版Uリブのすみ肉溶接部に発生するき裂の探傷を模擬したものである\cite{Urib1},\cite{Urib2},\cite{Urib3}.溶接部の複雑な形状がエコー形成に与える影響を調べ,どのような送受条件で有用なきずエコーが観測可能であるかをこのシミュレーションを通じて明らかにする.
最終,第5章では本研究で得られた知見をまとめ,結論と今後の課題を示す.なお,本研究では計算効率の制約のため全ての数値シミュレーションを2次元問題として行う.


\chapter{超音波擔傷試験の数値シミュレーション法}
	\chapter{超音波擔傷試験の数値シミュレーション法}
\section{問題設定}
図\ref{fig:}に、本研究で想定するT字継手における超音探傷試験の状況設定を示す。
継手は、水平部材に対して直角に近い角度で板材が溶接されているとし、
以下、便宜上、水平部材をウェブ、鉛直向きの板材をフランジと呼ぶ。
溶接部の詳細はこの図には示していないが、片側(右側)からの隅肉溶接が行われているとし、
継手右側を外側、左側を内側と称する。
なお、次章で示す数値シミュレーション結果では、比較のために、溶接部の余盛りを無視し、
ウェブとフランジが溶接金属が追加されることなくそのまま結合されたモデルや、
厚みが一定の平板に対する結果も示す。
き裂は、継手内側の角部から発生してウェブ側へ鉛直に近い方向へ伸びるものを考える。
現実には、継手に生じる応力場によってき裂の進展方向は変化するため、起点が同じでも
溶接金属内やウェブ側にき裂が伸びるケースもある。それらは、ウェブ側からの探傷が
必要で、超音波の送受信や伝播状況が図のものと大きくことなるため、別途、考察の
対象とする必要があるため、本研究では扱わない。
超音波の送受信は、継手外側のフランジ上側の面でのみ実施可能とする。
従って、送信、受信方向はき裂に対して強く制約され、検査に利用できる超音波エコーは
き裂からの後方散乱波だけとなる。もし、
送受信をき裂を挟むように、継手左右両側から行うことができるならば、
よく知られたTOFD法(time-of-flight diffraction)が利用できる。
ここでは、TOFDが適用できない状況がより困難で、詳細な検討を要する問題であるため、
片側からの探傷を扱う.
超音波の送信は、鋼橋の超音波探傷で通常用いられる圧電探触子で、フランジ上面から
行う。ただし、数値シミュレーションでは、圧電素子を直接モデル化することは避け、
既知の鉛直な表面力分布で与える。この方法では、表面力の時間変化に位相差が内場合、
探触子からの垂直入射を、直線的な位相差をつけることで斜角入射を模擬することができる。
受信は無指向性の点レシーバにより、継手外側の任意の範囲で行うことができるとする。
これは、波長に対して十分に小さいアレイ探触子での受信を想定したものである。
実際の探傷は、必ずしもアレイセンサーを用いる場合ばかりでは無いが、
アレイセンサーで受信した波形群を適切な遅延時間を設けて重ね合わせることで、
斜角探触子による受信も模擬できるため、この意味で、今回の設定は、一般のセンサー
による受信波形を模擬できる点でより包括的なものと言える。
\section{波動伝播問題の定式化}
鋼橋の超音波探傷試験に用いる超音波の周波数帯は、およそ2MHzから5MHzで、
その振幅は数十nm程度と非常に小さい。そのため、超音波伝搬は微小変形問題で、
媒体は線形弾性体とみなした解析を行うことができる。
このとき、弾性波動問題の支配方程式は、応力$\sigma_{ij}$, 速度$v_i$,
質量密度$\rho$を用いて
\begin{equation}
	\sigma_{ji,j}+f_i=\rho \dot{v}_i,
	\label{eqn:}
\end{equation}
ただし,$(,)$は空間の$\dot{()}$は時間に関する微分
\begin{equation}
	(\cdot)_{,i}=\frac{\partial (\cdot)}{\partial x_i}, \ \ 
	\dot{(\cdot)}=\frac{\partial (\cdot)}{\partial x_i}, \ \ 
	\label{eqn:}
\end{equation}
をそれぞれ表す.
2次元問題を考えるため、インデックス$i,j$は1または2で、総和規約を用いている.
速度$v_i$は変位$u_i$の時間微分であり、ひずみテンソル$\varepsilon_{ij}$は、
変位を使って
\begin{equation}
	\varepsilon_{ij}=\frac{1}{2}(u_{i,j}+u_{j,i})
	\label{eqn:}
\end{equation}
で、応力テンソルは弾性係数テンソル$C_{ijkl}$として
を用いて次のように表される。
\begin{equation}
	\varepsilon_{ij}=C_{ijkl}\varepsilon_{kl}
	\label{eqn:}
\end{equation}
ただし、ここでは媒体は等方性体とみなし、
\begin{equation}
	C_{ijkl}=\lambda \delta_{ij}\delta_{kl} +\mu \delta_{ik}\delta_{jl}
	\label{eqn:}
\end{equation}
とする。ここに、$\lambda, \mu$はラメ定数を表す。ラメ定数と密度より、
縦波と横波の位相速度は、順に
\begin{equation}
	c_{P}=\sqrt{\frac{\lambda + 2\mu}{\rho}}
	, \ \ 
	c_{S}=\sqrt{\frac{\mu}{\rho}}
	\label{eqn:}
\end{equation}
と与えられる.工学でしばしば用いられるヤング率$E$,ポアソン比$\nu$,せん断剛性$G$を
用いれば、ラメ定数は
\begin{equation}
	G=\mu, 
	\label{eqn:}
\end{equation}
$\mu$はせん断合成に一致している。
以上の式を与えられた初期条件、境界条件の元で解くことで、超音波の伝播シミュレーションを行うことができる。
\section{数値解析法(FDTD法)}

\chapter{シミュレーション結果}
	%\chapter{シミュレーション結果}
この章では、典型的な数値シミュレーション結果を用い,板や継手内部をどのように
超音波が伝播するかを明らかにする。はじめに、最も基本的なケースとして板内部と表面で
の波動伝播挙動を示し、モード変換や散乱波の発生状況を調べる。
次に、T字継手モデルに対して得られた解析結果を示し,継手部で部材が領域が分岐する
ことの効果がどのように現れるかを調べる。
これらの結果が超音波探傷試験の観点からどのような意味を持つかについては、
次章で、より複雑な形状のモデルを用いたシミュレーションの結果を踏まえて議論する。
\section{平板内の波動伝播}
解析モデルを図\ref{fig:fig3_00}に示す.
このモデルは表面き裂を持つ,十分に大きな板を想定したもので、
その70mmの範囲を解析対象としている。板両端部の打ち切り位置外側には
PML(perfectly matched layer)吸収領域を設け無反射条件を課す.
板上下面では、超音波の送信位置以外では応力フリー(トラクション$\fat{t}=\fat{0})$の
境界条件を与えた.入射波は平板上面の幅1mmの範囲(4.5$\leq x \leq $5.5)mmに加えた
一様な鉛直力て励起した。鉛直力の時間変化はガウス分布で振幅変調した余弦波で与え、
その周波数は5MHzとした.き裂の水平位置は$x=-12mm$とし,長さ4mmの鉛直き裂とした.
%--------------------
\begin{figure}[h]
	\begin{center}
	\includegraphics[width=0.7\linewidth]{Figs/model_plate.eps} 
	\end{center}
	\caption{
		鉛直表面き裂を有する平板モデル
	} 
	\label{fig:fig3_00}
\end{figure}
%--------------------
\subsection{入射波の挙動}
図\ref{fig:fig3_1}に,き裂が存在しない場合に生じる波動場、すなわち、入射場の様子を示す.
この図は、速度場$\fat{v}(\fat{x},t)$の5つの時刻におけるスナップショットを示したものである.
横軸は$x$を、縦軸は$y$座標を表し,各点での粒子速度$\fat{v}(\fat{x},t$の絶対値$|\fat{v}|$を
カラーマップとして表示したものである。各々の図には、送信時からの経過時刻が示してある。
これらのスナップショットにある波面のうち明瞭なものについては、伝播モードを
アルファベットで示している.これらの文字は、
Pが縦波を、Sが横波、Hがヘッド波,Rが表面波を表す.
(a)にあるように、鉛直力を表面に加えたことで、P波が鉛直方向に強い振幅を持って励起されて
最も早く進展し、その後ろにS波が続いている。
これらの波面形状はP波、S波とも同心円状だが指向性(放射パターン)は互いに異なる.
S波はおよそ45度の方向に強く、鉛直方向では弱い.逆にP波は、
外力の向きである鉛直下向きには強く放射されているが、水平方向に向かうについれて弱くなる.
P波とS波の波面を結ぶ直線的な波面は,ヘッド波呼ばれるモードである.
これらの波が進展すると、(b)にあるように板表面近傍に発生する表面波(レイリー波)が、
S波と分離して見られるようになる。これら4種類の波は、半無限領域表面に鉛直力を加えたときに
発生する波と同じもので、Lambの問題の解として理論的に存在と挙動が説明されている。
P波は、(b)に示した時刻では板の下面に達して反射し,P波とS波の両方を発生させる。
図ではこれらの反射で生じたP波とS波を,それぞれ,P-P,P-Sとして示している。
P波は鉛直入射した場合を除き,P,S両方のモードを発生させ,
P-Pの後を伝播速度の遅いP-S波が追いかける形になっている。
(c)では、S波が下面に達した直後の状況を示している。S波についても、P、S両方の反射波
(S-PとP-S)が発生している。この様子は更に時間が経過した(d)の図でより明確であるため、
(d)ににみ,S-P波とS-S波の波面位置を示している。
なお、(c)ではP波が2回目の反射を板上面で起こし、振幅をあまり低下させることなく伝播している。
P波はS波のおよそ2倍の位相速度で進行するため、常にS波を追い越しながら伝播する.
(e)にもあるように、板内部では、P波、S波とも、モード変換を伴いつつ多重反射を起こす.
その結果、領域は単純な形をしているにも関わらず,板内部の波動場は
時間経過についれて非常に複雑化つ乱雑なものに見えてくる.
なお、このような多順反射波が十分な回数繰り返されて干渉を起こした結果がガイド波である。
ここでの計算条件では、P波の波長が約1.2mm、横波波長が約0.6mmと厚に比べて小さい.
そのため、ガイド波が形成されるまでには非常に長い時間が必要とされる.このことから、
超音波探傷で問題となる観測時間範囲においては、ガイド波としての解析や解釈
は有効でない.
%--------------------
\begin{figure}[h]
	\begin{center}
	\includegraphics[width=0.8\linewidth]{Figs/plate_inc.eps} 
	\end{center}
	\caption{
		速度場$|\fat{v}(\fat{x},t)|$のスナップショット.
		P,Sは縦波,横波を,Rは表面波,Hはヘッド波を表す.
		ハイフンで繋がれた文字は反射前後でのモードを表す. き裂が存在しない場合.
	} 
	\label{fig:fig3_1}
\end{figure}
%--------------------
\subsubsection{き裂を含む板内部の波動場}
次に,$x=-12$mmに深さ4mmの鉛直き裂が存在する場合の波動場の様子を図\ref{fig:fig3_2}に示す.
この図は図\ref{fig:fig3_1}と同様に、速度場$|\fat{v}|$のスナップショットを示したもので、
き裂が存在すること以外の計算条件は,前の図のものと同じである。
ここでは、き裂位置は白の実線で示してあり、き裂によって生じる散乱波の挙動が示されている.
鉛直力の付加によって励起された波動場は、き裂に到達するまでは図\ref{fig:fig3_1}と全く同じ
挙動を示す。従って(a)の結果は、図\ref{fig:fig3_1}(b)と同じである。図\ref{fig:fig3_2}(a)
では,最初のP波がき裂に到達する直前の状態を示している。この後、P波の一部はき裂で
散乱され、右方向に散乱波が発生する。ただし、散乱波は入射波に比べて小さく、
同じスケールで描画した図\ref{fig:fig3_2}では、いつ励起されてどのように伝播するかがわかりにくい.
このことから、図\ref{fig:fig3_2}では、き裂に到達する主要な波が、各々の時刻に
どの位置にあるかまでを示している。
例えば、下面で一回反射した後のP波(P-P波)は、(b)の時刻にはき裂下側から近づいており、
その下にモード変換で生じたP-S波が続いている。P-S波は(d)の図では、き裂先端部に達しており、その直後
にレイリー波(R波)が、き裂のコーナー部分(開口部)に到達していることが(e)の図に示されている.
\begin{figure}[h]
	\begin{center}
	\includegraphics[width=0.8\linewidth]{Figs/plate_tot.eps} 
	\end{center}
	\caption{
		速度場$|\fat{v}(\fat{x},t)|$のスナップショット.
		P,Sは縦波,横波を,Rは表面波,Hはヘッド波を表す.
		き裂がある場合.
	} 
	\label{fig:fig3_2}
\end{figure}
\subsubsection{散乱波の発生と伝播}
散乱波$\fat{v}^{sc}$は、き裂が存在する場合に生じる波動場$\fat{v}$と、
き裂が存在しない場合の波動場$\fat{v}^{in}$の差として、
\begin{equation}
	\fat{v}^{sc}(\fat{x},t)=\fat{v}(\fat{x},t)-\fat{v}^{in}(\fat{x},t)
	\label{eqn:def_vsc}
\end{equation}
で定義される。
この定義より、き裂がある場合の波動場$\fat{v}$は、
$\fat{v}^{in}+\fat{v}^{sc}$と、入射場と散乱場の和で表されることから、
$\fat{v}$は全波動場、あるいは、より具体的に全速度と呼ばれることがある。
これに対して、$\fat{v}^{in},\fat{v}^{sc}$は、入射波成分、散乱波成分などと呼ばれる。
このような呼び方をするき、図\ref{fig:fig3_1}に示された結果は
全て入射波成分とみなす。従って、入射波には、板表面で反射やモード変換した
P波やS波も含まれる。
実験では、$\fat{v}$と$\fat{v}^{in}$をそれぞれ
求めることは難しいが、数値シミュレーションでは既に示したように
き裂の有無のみを変えたモデルで同じ計算を行うことによって、
$\fat{v}$と$\fat{v}^{in}$を別々に求められる.
そこで、式(\ref{eqn:def_vsc}に従って散乱波$\fat{v}^{sc}$を求め、
その結果をスナップショットとして示したのが図\ref{fig:fig3_3}である。
この図は、図\ref{fig:fig3_2}と同じ5つの時刻における散乱場の様子を
示したものである。ただし、散乱波は入射波に比べて弱いため、
カラー表示する際のスケールは変えてある。
図\ref{fig:fig3_3}(a)の時点では、入射波はき裂に到達していないため
散乱波は生じていない。(b)は、最初のP波が到達した直後で、
3種類のモードの散乱波が現れている.
このうちP1は右方向に伝わる縦波散乱波で、入射波の到来方向に戻ることから
後方散乱波と呼ばれる。
同様に、き裂を超えて左方向に進む散乱波前方散乱波と呼ばれる。
前方散乱波は、本研究の問題設定では観測できないものであることから
以下では議論の対象としない。
S1は、横波の散乱波で、特徴的な円弧状の波面を形成している。
これは、き裂先端で発生した回折波であることから、端部エコーと呼ばれる。
これら縦波後方散乱波と横波端部エコーを結ぶように、わずかにヘッドウェーブ(H1)が現れている。
端部エコーは、き裂先端位置の情報を持つことから、
超音波探傷試験では重要な意味をもつ。

図\ref{fig:fig3_3}(c)は、P-P波によって発生した散乱波(P2,H2)が見られる。
P2とH2は、右下がりのき裂の長さと同程度の波面を示している。
これは、右下方向から入射したP-P波が、き裂端部で回折して発生した
散乱波で、丁度板の上面に到達した時刻にあたる。
これらP2,H2はき裂開口部のコーナでS波に展示、(d)にあるように、板表面方へS2とS3'が、
き裂表面にそって下向きにS3が伝播している。
その後S3は、き裂から離れて伝播を続けることが(e)に示さている。
なお、(e)の時刻は、強い表面波(R)がき裂面に到達した時刻で、この後、顕著な後方散乱波が生じる。
その挙動については次の章であらためて議論する。
%
\subsubsection{走時波形}
超音波探傷試験では、試験体表面の振動が計測可能で、内部の波動場を直接観測
することはできない。そこで、
\begin{equation}
	0\leq x \leq 45 , \ \ y=12[{\rm mm}]
	\label{eqn:Rapt}
\end{equation}
の観測領域で得られた波形がのようであるかを検討する。粒子速度$\fat{v}=(v_1, v_2)$は
ベクトル量で,実験では水平成分$v_1$と鉛直成分$v_2$いずれを測るセンサーも開発されている。
ただし、試験片表面方向の成分である$v_1$は、通常、鉛直成分$v_2$の計測よりも難しく、
圧電センサーの多くは厚み振動のモードを利用して鉛直動を得るものである。
そこで、鉛直成分$v_2$だけが取得できるものとしてここでの議論を進める。
図\ref{fig:fig3_4}は、式(\ref{eqn:Rapt}の観測領域で得られた鉛直方向速度$v_2(\fat{x},t)$
の走時波形である。このプロットは横軸に時間$t$、縦軸に位置$x$をとり、
速度成分$v_2$をカラー表示したものである.超音波探傷分野では、このような波形表示方法を
Bスコープ、あるいはBスキャン波形と呼ぶ.
図\ref{fig:fig3_4}の(a)は、き裂の無い平板で計算した$v_2$、すなわちの入射波成分を、
(b)はき裂が存在する場合の全速度成分を示す。一方、(c)は(b)と(a)の差分として
得られる散乱波成分である.超音波探傷において密に多点計測を行った結果として得られるは、
(b)の全波動場で、 このようなデータからき裂の有無や位置を推定することが求められる。
走時波形に現れる右上がりの曲線は、位置に対して到達時間が遅れることを示し、
右($x>0$)方向への進行波であることを意味する。
図\ref{fig:fig3_4}にはほとんど含まれないが、左上がりの曲線はこの反対で、
左方向$x<0$への進行波を表す.
図\ref{fig:fig3_4}(a)では、30$\mu$sを過ぎることからは、ほとんど鉛直に近い
直線が繰り返す。これは、縦波がどの位置でもほぼ板厚方向へ繰り返し往復して、
定在波に近づくことを示している.
一方、10$\mu$s程度までの早い時間帯では、鉛直方向から傾いた曲線や直線が
現れる。一番はやく現れるのは、板表面に沿って伝播するP波で、この後続く、
非常に大きな振幅を持った直線(R)は表面波である.
これらの波は、板表面に沿って伝わるため、直線の勾配が位相速度を与える。
P波とR波の後には、P-P波が続くが、これ以後のP-S波は、S-P波とも重なる箇所が
あり、走時波形だけから経路を特定することは難しくなる.
き裂がある場合は、(a)に散乱波成分が加わったものとなるため、
基本的なパターンは似たものとなる。しかながら、(c)に示す散乱波成分の中
には非常に強いものもあり、例えばR-Rと示した直線は
表面波によって励起された後方散乱波が、表面波として伝わった結果として現れる
ものである。このような後方散乱波は、この後示す他のモデルでの解析でも
大きな振幅を持った散乱波として普遍的に現れる.
なお、図\ref{fig:fig3_4}(c)のような散乱波成分のみの走時データは、実験では得られない
ことに注意が必要である。しかしながら、数値シミュレーションによって散乱波の挙動を調べて
置けば、実験において目的とする散乱波のモードに合わせて計測位置や計測時間範囲を設定できる。
また、図\ref{fig:fig3_1}-\ref{fig:fig:3_3}のような 結果を参照すれば、走時曲線に現れる
散乱波がどのような入射波モードで励起され、どのような経路を経て到来したかも理解
できる場合が多い。入射波と対応する散乱波の伝播経路が特定できればき裂位置の推定が可能で、
き裂端部の位置を調べることは非破壊検査の観点から特に重要である。
\begin{figure}[h]
	\begin{center}
	\includegraphics[width=0.8\linewidth]{Figs/plate_sct.eps} 
	\end{center}
	\caption{
		速度場$|\fat{v}(\fat{x},t)|$のスナップショット.
		P,Sは縦波,横波を,Rは表面波,Hはヘッド波を表す.
		き裂がある場合の散乱波成分のみを表示.
	} 
	\label{fig:fig3_3}
\end{figure}
\begin{figure}[h]
	\begin{center}
	\includegraphics[width=0.8\linewidth]{Figs/plate_bwvs.eps} 
	\end{center}
	\caption{
		$x=0\sim35, y=12$mmの位置で得られた波形の走時プロット.
		(a)き裂なし,(b)き裂ありの場合.
		(c)はき裂ありと無しの場合の差分をとって計算した散乱波成分.
	} 
	\label{fig:fig3_4}
\end{figure}
%--------------------
\section{T継手における波動伝播挙動(余盛り形状を考慮しない場合)}
%--------------------
図\ref{fig:fig3_01}にT継手部の解析モデルを示す。
フランジは板厚12mm、ウェブの板厚は8mmとし、両者が直角に接合されたT継手の
モデルを考える。領域が分岐することの効果を調べるため、継手ディテールは単純化し、
余盛りやルートギャップなどの形状は考慮しない。き裂と入射波の送信方法は
比較ができるよう前節の平板と同じにする。すなわち、き裂水平位置$x=-12$mm, 
深さ4mm、入射波は$x=5$mmの幅1mmの範囲に加えた等分布荷重で励起する。
荷重の時間変化も前節と同じように設定し,5MHZの波を入射する。
%%
\begin{figure}[h]
	\begin{center}
	\includegraphics[width=0.7\linewidth]{Figs/model_T.eps} 
	\end{center}
	\caption{
		単純化したT継手モデル
	} 
	\label{fig:fig3_01}
\end{figure}
%--------------------
\subsection{入射波の挙動}
はじめに、き裂が無い場合の計算を予め行い入射場$\fat{v}^{in}$を求めた。
この結果の一部を、速度場$\left| \fat{v}^{in}\right|$のスナップショットとして
図\ref{fig:fig3_5}に示す。
この図の(a)は、送信後、最初のP波が継手に達した瞬間を示している。
この時間までは平板と同じであるため、P,S,H波が現れている。
この後、(b)の図にP波、H波は継手右側のコーナで回折波を発生させ、ウェブ側にも
波動場が進展するが、フランジ側の伝播挙動は板の場合とあまり違いがない。
次に、(c)から(d)の時間では、継手に達した表面波とS波がコーナー部分で回折を
起こしながら継手内部に親友している。表面波は部材内部では存在できないため、
表面波はS波となって継手内を進み、(e)の時刻で丁度左側のコーナーに到達する。
なお、フランジの底面で反射したP-P波は、フランジ板厚と縦波位相速度の兼ね合いから、
S波とほとんど同じときに継手右コーナーに達し、その後ウェブ側に入る(図\ref{fig:fig3_5}(d))。
この後、(e)と(f)に見られるように、P-P波はウェブ表面で反射し、新たにP-P-P波、P-P-S波
を励起している。なお、継手内部ではR-S波となった成分は、継手部分を通過後は
再度R波となって進んで行く。(f)の図にはこの様子が僅かながら現れている。
\begin{figure}[h]
	\begin{center}
	\includegraphics[width=1.0\linewidth]{Figs/T_inc.eps} 
	\end{center}
	\caption{
		速度場$|\fat{v}(\fat{x},t)|$のスナップショット.
		P,Sは縦波,横波を,Rは表面波,Hはヘッド波を表す.
		ハイフンで繋がれた文字は反射前後でのモードを表す. き裂が存在しない場合.
	} 
	\label{fig:fig3_5}
\end{figure}
\subsubsection{き裂を含む場合の波動場}
T継手の脚部にき裂がある場合のシミュレーション結果を図\ref{fig:fig3_6}に示す。
この図は、全速度成分$\fat{v}$を示したもので、散乱波成分が含まれる。
散乱波成分は既に述べたように、入射波に比べて非常に小さく同じスケールでは
分布や進展挙動が見えにくい。そこで、図\ref{fig:fig3_6}では、代表的な
モードの波が、散乱波を生ずる前後にどのような状態にあるかを指摘する。
図\ref{fig:fig3_6}(a)は、最初のP波がき裂で散乱された直後の状態を示す。
このときには、他の入射波成分はき裂がない状態と全く同じである。
(b)はこの後、P-P波がき裂に対して下から伝播し、最初にき裂端部に到達する
瞬間の様子である。P-P波はき裂面で反射を起こしつつ(c)の時間にはウェブ内部に進んでいる。
この次の時刻(d)では、R-S波とP-S波がほぼ同時にき裂へ達しつつある。
(e)の図では、R-S波はき裂によってほぼ完全に進路を遮断され、後方散乱波に転じている。
P-S波はき裂端部に到達した後は、(e)と(f)の図にあるようにウェブ側へ進む。
平板の場合との違いは、当然ながらP-P波やP-S波といった反射はがウェブ内に進行する
点で、板上限面での多重反射波が繰返しき裂で散乱されることが無い点にある。
\begin{figure}[h]
	\begin{center}
	\includegraphics[width=1.0\linewidth]{Figs/T_tot.eps} 
	\end{center}
	\caption{
		速度場$|\fat{v}(\fat{x},t)|$のスナップショット.
		P,Sは縦波,横波を,Rは表面波,Hはヘッド波を表す.
		き裂がある場合.
	} 
	\label{fig:fig3_6}
\end{figure}
%%
\subsubsection{散乱波の発生と伝播}
全速度$\fat{v}$から入射波成分$\fat{v}^{in}$を差し引いて求めた、散乱場$\fat{v}^{sc}$の
進展挙動を図\ref{fig:fig3_7}に示す。図\ref{fig:fig3_7}は図\ref{fig:fig3_6}に示した
結果と同じ時刻の散乱波成分の分布を示したものである。ただし、散乱波成分を見えやすく
するため、カラーマップのスケールは互いに異なっている。(a)と(b)の時刻では、
最初にき裂へ直接到達するP波で生じた散乱波P1とS1が見られる。
P1は後方散乱波となって入射方向に戻るものが強く、S1波はほぼ全円状の波面となって
おり、き裂端部エコーとして発生したことが分かる。
この後の時刻(c)では下からのP−P波が縦波P2と、横波散乱波S2を発生させている。
P2、S2とも、き裂面での反射とき裂端部での回折波から構成されている様子が(c)の図から分かる。
この後(d)の図でP2はウェブ内へ入るが、(f)の時刻にかけてS2はウェブ脚部のコーナに
向かう。S2はコーナに到達後再び散乱されるが、(e)の時刻に発生したR-Sに起因する強い横波S3
の影響が強く(f)の時点でその状態を確認することは出来ない。
以上のことから、T継手の場合、P波、P-P波、R-S波の順に散乱波を励起するが、
P-P波はき裂での散乱の後、ウェブの方向に進行するため、フランジ内を繰返し往復するような
散乱波は生じ得ないことが分かる。つまり、継手形状は平板よりも複雑だが、
平板の場合のような周期的な散乱波の発生が少なく、散乱波の伝播挙動はむしろ理解しやすい
ものになっている。
\begin{figure}[h]
	\begin{center}
	\includegraphics[width=1.0\linewidth]{Figs/T_sct.eps} 
	\end{center}
	\caption{
		速度場$|\fat{v}(\fat{x},t)|$のスナップショット.
		P,Sは縦波,横波を,Rは表面波,Hはヘッド波を表す.
		き裂がある場合の散乱波成分のみを表示.
	} 
	\label{fig:fig3_7}
\end{figure}
\subsubsection{走時波形}
図\ref{fig:fig3_8}に、フランジ上面で観測した鉛直動$v_2(\fat{x},t)$の走時波形を示す。
この図にある3つのプロットは、上から順に(a)入射波成分、(b)き裂がある場合の全速度、
(c)散乱波成分を示したmのである。入射波には平板の対応する結果である図\ref{fig:fig3_4}(a)と
類似したパターンになっている。一方、散乱波成分(c)についてみると、目立つものは
10$\mu$sec付近から伸びる表面波の伝播を示す直線(S3-R)しかなく、平板の場合よりも
大きな振幅をもって観測点に到達するエコーが少ないことが分かる。
このS3-Rは、図\ref{fig:fig3_7}(f)にある、横波S3が、フランジ表面で表面波に転じて
長距離を減衰することなく進行するものであることは、各点での到達時間から明らかである。
\begin{figure}[h]
	\begin{center}
	\includegraphics[width=0.8\linewidth]{Figs/T_bwvs.eps} 
	\end{center}
	\caption{
		$x=0\sim35, y=12$mmの位置で得られた波形の走時プロット.
		(a)き裂なし,(b)き裂ありの場合.
		(c)はき裂ありと無しの場合の差分をとって計算した散乱波成分.
	} 
	\label{fig:fig3_8}
\end{figure}
%--------------------



\chapter{継手ディテールを考慮した超音波伝播シミュレーション}
	%\chapter{継手ディテールを考慮した超音波伝播シミュレーション}
この章では、T継手の詳細な形状を考慮したモデルを用いた超音波伝播解析を行う.
解析結果は、前章で示したより基本的なモデルでの結果をベースとして解釈する
とともに、超音波探傷試験の観点から見た意味について議論する。
ここでも、解析結果は、き裂の無いモデルで計算した入射波動場、
き裂があるモデルでの全波動場、両者の差として得た散乱波動場の順に示す。
その後、走時波形を示したのち、走時波形から検出できる散乱波成分が
どのようにして生じたものか、その超音波探傷試験における有用性について
議論を行う。
\section{解析モデル}
解析モデルの形状を図\ref{fig:4_0}に示す。
このモデルは、鋼床版Uリブの隅肉溶接部に生じた疲労き裂の超音波探傷を
想定したものである。フランジはデッキプレートに、ウェブはUリブの一部
を模擬したもので、ウェブはフランジに対して15度直交方向から
傾いている。き裂はルートギャップからデッキプレート側へ進展した場合の
モデルとするため、ウェブ左側脚部に1mm角のギャップを設け、
その端部からき裂を設けてある。また、ウェブ右側は隅肉溶接の余盛り
を表現するため、脚長7mmの余盛りとなるよう、曲率半径20mmの円弧で
囲われた領域を設けてある。
入射波の励起は、水平位置$x=5$mm, 幅1mmの区間に加えた一様な
鉛直荷重で、これまでの計算と同じ条件で行う。
以下、これら継手ディテールが、超音波伝播にどのような影響を与えるかという
点を中心にして、解析結果を検討する。
\begin{figure}[h]
	\begin{center}
	\includegraphics[width=0.7\linewidth]{Figs/model_bead.eps} 
	\end{center}
	\caption{
		隅肉溶接の余盛りを考慮した詳細なT継手モデル
	} 
	\label{fig:fig4_0}
\end{figure}
%--------------------
\begin{figure}[h]
	\begin{center}
	\includegraphics[width=1.0\linewidth]{Figs/bead_inc.eps} 
	\end{center}
	\caption{
		速度場$|\fat{v}(\fat{x},t)|$のスナップショット.
		P,Sは縦波,横波を,Rは表面波,Hはヘッド波を表す.
		ハイフンで繋がれた文字は反射前後でのモードを表す. き裂が存在しない場合.
	} 
	\label{fig:fig4_1}
\end{figure}
\subsection{入射波の挙動}
図\ref{fig:fig4_1}に、入射場の進展を速度場$|\fat{v}^{in}(\fat{x},t)|$の
スナップショットとし示す。入射波の様子は(a)の時点では、T継手の場合と大差なく、
P波が止端部で回折を起こした後、溶接部に進む。(a)は、表面波が止端部に
達した時点の様子を、(b)では表面波に起因した回折波が生じて余盛り表面に沿う
方向へ表面波の一部が進行している様子を示す。さらに時間が経過すると、(c)では
P-P波が余盛り部に侵入し、(d)では余盛り表面で強い反射波P-P-PとP-P-Sが生じる.
縦波反射波であるP-P-Pは、ウェブに達した後(e)にあるようにウェブ左側面へ、
横波P-P-Sはルートギャップに向かって進むことが(e)の図から分かる。
P-P-Sの波面は余盛りの形状を反映してレンズ状になっており、次第に収束する.
その結果(f)の時点では多重反射を繰り返す縦波と同程度の振幅を持つに至っている。
このような挙動は前章で用いた単純なT継手では現れず、余盛り形状の効果を
反映したものであることが明らかである。
\begin{figure}[h]
	\begin{center}
	\includegraphics[width=1.0\linewidth]{Figs/bead_tot.eps} 
	\end{center}
	\caption{
		速度場$|\fat{v}(\fat{x},t)|$のスナップショット.
		P,Sは縦波,横波を,Rは表面波,Hはヘッド波を表す.
		き裂がある場合.
	} 
	\label{fig:fig4_2}
\end{figure}
\subsection{き裂が存在する場合の波動場}
き裂を含む場合の全波動場の様子を\ref{fig:fig4_2}に示す。
ここでも、スナップショットとして表示する時刻は、代表的な入射波が
き裂に達する前後となることを基準に選んである。
(a)では、最初のP波がき裂に入射した直後で、前章のT継手モデルとほとんど同じ状況が
再現されている。(b),(c)に関しても概ね同様で、ウェブがフランジに対して傾いては居るが、
P-P波や、R-S波が継手内部を波面の形状を大きく乱すことなく進んでいる。
(d)からはレンズ状のP-P-Sの発生がこれまでとことなるが、R-S波は(d)から(e)の時点で
き裂に到達して後方散乱波を発生させている。(f)では、収束して大きな振幅をもった
P-P-Sがき裂近傍に到達している。
\begin{figure}[h]
	\begin{center}
	\includegraphics[width=1.0\linewidth]{Figs/bead_sct.eps} 
	\end{center}
	\caption{
		速度場$|\fat{v}(\fat{x},t)|$のスナップショット.
		P,Sは縦波,横波を,Rは表面波,Hはヘッド波を表す.
		き裂がある場合の散乱波成分のみを表示.
	} 
	\label{fig:fig4_3}
\end{figure}
\begin{figure}[h]
	\begin{center}
	\includegraphics[width=1.0\linewidth]{Figs/bead_sct2.eps} 
	\end{center}
	\caption{
		速度場$|\fat{v}(\fat{x},t)|$のスナップショット.
		P,Sは縦波,横波を,Rは表面波,Hはヘッド波を表す.
		き裂がある場合の散乱波成分のみを表示.
	} 
	\label{fig:fig4_4}
\end{figure}
\subsection{散乱波の発生と伝播}
き裂を含むモデルでの散乱波成分を抽出した結果を図\ref{fig:fig4_3}に示す。
ここに示した結果の内、(a)から(d)は、継手ディテールを考慮しない
前章のモデルで得られたものとほとんど同じであることが分かる。
一方(e)と(f)の時刻では、P-P波からき裂での散乱によってS波に転じたS2波が、
余盛りに向かって進行し、直接ウェブに入らない点がこれまでと異なっている。
また、継手を横断したR-S波に起因するS3後方散乱波は、簡易T継手モデルの場合
よりもやや振幅が小さくなっている。
この後に示す走時波形には、これらS2、S3波が大きな振幅をもって観測される
と予想されるため、ここでは、図\ref{fig:fig4_3}(f)の後の状況についても
見ておく。

図\ref{fig:fig4_4}は、図\ref{fig:4_3}に続く散乱波動場の様子を
0.5$\mu$sおきに示したものである。強いS3後方散乱波は、スピードが遅く
ゆっくりと右方向に波面と振幅をほとんど変化させずに進んでいる。
余盛り表面を経由してフォーカスのかかったS2波は、(a)の時間に
き裂上端部にあたり、(b)ではその結果モード変換した散乱波S2-Pが現れている。
この縦波は、(c)の図では右下に向かってき裂から離れていく様子が見られる。
その後ろには、同時に発生した横波散乱波S2−SがS2−Pと分離して現れる。
この横波散乱波S2-Sは大きな振幅を持ち、S2-Pよりも狭い範囲に集中したまま
進んでいく。これら,S2-P,S2-Sは、いずれもフランジ底面で反射して、
各々フランジ上面に向けて進むP波とS波を生じさせる。
これらは、余盛り表面が音響レンズとして作用した結果として生じたもので、
余盛り形状が既知であれば、非破壊検査に利用できる可能性のある成分としての
重要性がある。

\subsection{走時曲線}
最後に、走時曲線を図\ref{fig:fig4_5}に示す。
入射波、全波動場、散乱波を上から順に示すことは、これまでと同様である。
ここでも、入射波とき裂が存在する場合の全波動場の走時曲線は、単純なT継手
の場合と大差ない。一方で散乱波成分を抽出してみると、表面波(S3-R)が
非常に大きな振幅をもって伝わることはこれまで通りだが、
余盛りのレンズ効果に起因した散乱波がこのモデルの場合にのみ現れている。
このように、散乱波の走時曲線に現れるということは、入射波成分の分離という
課題は残るにせよ、実際に観測可能性があることを示唆し、観測が実現すれば
超音波探傷にも利用できると考えられる.
\begin{figure}[h]
	\begin{center}
	\includegraphics[width=0.8\linewidth]{Figs/bead_bwvs.eps} 
	\end{center}
	\caption{
		$x=0\sim35, y=12$mmの位置で得られた波形の走時プロット.
		(a)き裂なし,(b)き裂ありの場合.
		(c)はき裂ありと無しの場合の差分をとって計算した散乱波成分.
	} 
	\label{fig:fig4_5}
\end{figure}
%--------------------



%--------------------
\chapter{結論と今後の課題}
本研究では,T継手隅肉溶接部の超音波探傷試験を,より信頼性の高いものとしていくことを目的に,超音波エコーの励起と伝搬メカニズムについて数値シミュレーションで詳細な検討を行った.
ここでは,平板と溶接部の形状を無視した簡易なT継手モデルを用いた波動伝播解析により,継手周辺での超音波の基本的な伝播と鉛直き裂による散乱挙動を調べた.
その結果を踏まえ,溶接部の形状を詳細にモデルかした継手モデルでのシミュレーションを行い,超音波探傷の実施において有用となる以下の知見を得た.
\begin{enumerate}
\item
	継手部分に向かう入射波の挙動は,平板内の波動伝播挙動に概ね同じで,
縦波と横波の反射およびモード変換の繰り返しと,横波から分離した表面波の伝播として理解できる.
\item
	溶接止端部に入射波が到達すると回折波が発生するが,回折波が継手内部で
	大きな振幅をもつことは無く,き裂エコーの形成にもほとんど影響しない.
\item
	溶接止端部を超えた表面波は,S波となって継手内をき裂に向けて直進する.
	き裂への到達後は強い横波の後方散乱波を発生させる.このようにして発生した
	横波は継手外部で再び表面波となり送信側に戻る.
\item
	表面波として送信側に戻る上記のエコーは,走時波形に顕著に現れる.そのため,
	被検査材の表面を表面波が伝播可能な状況である場合,き裂の検出と
	水平位置の同定に有用となる.
\item
	き裂に最も早く到達する縦波は,縦波後方散乱波として最初に送信側に戻る.
	その一部である端部エコーは,フランジの下面を経由して観測領域に到達する.
	これらはき裂の検出と位置の同定に利用できるが,振幅が小さいことから,
	高精度な計測や入射波との分離が課題となる.
\item
	余盛り形状が凸なとき,余盛は,継手内部で横波を集束させるレンズの効果を果たす.
	その結果,き裂で生じた散乱波は探傷波形(走時波形)中に認められ,き裂の探傷に
	利用できる可能性がある.
	ただし,その実現のためには予め余盛り形状を計測し,レンズ効果による波動場の
	収束状況を把握しておくことが前提となる.
\item
	今回検討した探傷条件は制約が厳しく,観測しうるき裂エコーの種類も少ない.
	従って,上記の表面波,縦波散乱波,余盛りのレンズ効果に起因した
	横波散乱波の検出と利用により,検査モードを多重化することが,検査の信頼性を向上
	させるために必要と考えられる.
\end{enumerate}
今後は,本研究で得られた結論の実験的な裏付けを得ること,き裂向きや余盛り形状と
大きさの影響を数値シミュレーションによって更に詳しく調べることが課題となる.
また,本研究で得られた知見を踏まえ,斜角探触子やフェーズドアレイ探触子を使った
探傷における送受条件を最適化すること,ウェブ側からの計測が可能な場合の
き裂エコー経路の同定も重要な課題となる.
これらの課題を解決することで,継手に内在する種々のき裂を確実に捉えることのできる
超音波擔傷技術の構築につながる.その過程では,継手部の形状や想定しうる探傷条件が
が多様であることから,本研究で行ったような数値シミュレーションによる超音波伝播挙動の
把握が重要な役割を果たす.この点では,超音波伝播解析プログラムの更なる大規模化と高速化も,
シミュレーションツールとしての普及,実用化のために
取り組むべき課題の一つと言える.
\renewcommand{\bibname}{参考文献}
\begin{thebibliography}{99}
%\begin{spacing}{1.175}
	\bibitem{Miki}
	三木千壽: 鋼橋の疲労と破壊,朝倉書店,2011.

\bibitem{US}
	中村 僖良: 超音波,日本音響学会 編, コロナ社,2001. 
\bibitem{JDA}
	J.D.Achebach: Wave propagation in elastic solids, North-Holland, 1973.
\bibitem{Urib1}
	服部 雅史, 牧田 通, 舘石 和雄, 判治 剛, 清水 優, 八木 尚人:
	鋼床版Uリブ・デッキプレート溶接部の内在き裂に対するフェーズドアレイ超音波探傷の測定精度, 
	土木学会論文集A1(構造・地震工学), 2018, 74 巻, 3 号, p.516-530.
\bibitem{Urib2}
	八木 尚人, 鈴木 俊光, 若林 登, 村野 益巳, 三木 千壽: 
	鋼床版トラフリブ溶接部の内在疲労き裂に対するフェーズドアレイ超音波探傷試験の適用,
	土木学会論文集A1(構造・地震工学), 2016, 72 巻, 3 号, p. 393-406.
\bibitem{Urib3}
	村越 潤, 高橋 実, 小池 光裕, 木村 友則: 臨界屈折角近傍に調整した超音波斜角探触子による鋼床版デッキ進展き裂の探傷法の検討, 
	土木学会論文集A1(構造・地震工学), 2012, 68 巻, 2 号, p. 453-464
\bibitem{FDTD1}
	佐藤雅弘: FDTD法による弾性振動・波動の解析入門,森北出版株式会社,2003.
\bibitem{FDTD2}
	橋本 修: 実践FDTD時間領域差分法,森北出版株式会社, 2006.
%\end{spacing}
\end{thebibliography}
\end{document}
%%%%%%%%%%%%%%%%%%%%%%%%%%%%%%%%%%%%%%%%%%%%

