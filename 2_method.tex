\chapter{超音波擔傷試験の数値シミュレーション法}
\section{問題設定}
図\ref{fig:}に、本研究で想定するT字継手における超音探傷試験の状況設定を示す。
継手は、水平部材に対して直角に近い角度で板材が溶接されているとし、
以下、便宜上、水平部材をウェブ、鉛直向きの板材をフランジと呼ぶ。
溶接部の詳細はこの図には示していないが、片側(右側)からの隅肉溶接が行われているとし、
継手右側を外側、左側を内側と称する。
なお、次章で示す数値シミュレーション結果では、比較のために、溶接部の余盛りを無視し、
ウェブとフランジが溶接金属が追加されることなくそのまま結合されたモデルや、
厚みが一定の平板に対する結果も示す。
き裂は、継手内側の角部から発生してウェブ側へ鉛直に近い方向へ伸びるものを考える。
現実には、継手に生じる応力場によってき裂の進展方向は変化するため、起点が同じでも
溶接金属内やウェブ側にき裂が伸びるケースもある。それらは、ウェブ側からの探傷が
必要で、超音波の送受信や伝播状況が図のものと大きくことなるため、別途、考察の
対象とする必要があるため、本研究では扱わない。
超音波の送受信は、継手外側のフランジ上側の面でのみ実施可能とする。
従って、送信、受信方向はき裂に対して強く制約され、検査に利用できる超音波エコーは
き裂からの後方散乱波だけとなる。もし、
送受信をき裂を挟むように、継手左右両側から行うことができるならば、
よく知られたTOFD法(time-of-flight diffraction)が利用できる。
ここでは、TOFDが適用できない状況がより困難で、詳細な検討を要する問題であるため、
片側からの探傷を扱う.
超音波の送信は、鋼橋の超音波探傷で通常用いられる圧電探触子で、フランジ上面から
行う。ただし、数値シミュレーションでは、圧電素子を直接モデル化することは避け、
既知の鉛直な表面力分布で与える。この方法では、表面力の時間変化に位相差が内場合、
探触子からの垂直入射を、直線的な位相差をつけることで斜角入射を模擬することができる。
受信は無指向性の点レシーバにより、継手外側の任意の範囲で行うことができるとする。
これは、波長に対して十分に小さいアレイ探触子での受信を想定したものである。
実際の探傷は、必ずしもアレイセンサーを用いる場合ばかりでは無いが、
アレイセンサーで受信した波形群を適切な遅延時間を設けて重ね合わせることで、
斜角探触子による受信も模擬できるため、この意味で、今回の設定は、一般のセンサー
による受信波形を模擬できる点でより包括的なものと言える。
\section{波動伝播問題の定式化}
鋼橋の超音波探傷試験に用いる超音波の周波数帯は、およそ2MHzから5MHzで、
その振幅は数十nm程度と非常に小さい。そのため、超音波伝搬は微小変形問題で、
媒体は線形弾性体とみなした解析を行うことができる。
このとき、弾性波動問題の支配方程式は、応力$\sigma_{ij}$, 速度$v_i$,
質量密度$\rho$を用いて
\begin{equation}
	\sigma_{ji,j}+f_i=\rho \dot{v}_i,
	\label{eqn:}
\end{equation}
ただし,$(,)$は空間の$\dot{()}$は時間に関する微分
\begin{equation}
	(\cdot)_{,i}=\frac{\partial (\cdot)}{\partial x_i}, \ \ 
	\dot{(\cdot)}=\frac{\partial (\cdot)}{\partial x_i}, \ \ 
	\label{eqn:}
\end{equation}
をそれぞれ表す.
2次元問題を考えるため、インデックス$i,j$は1または2で、総和規約を用いている.
速度$v_i$は変位$u_i$の時間微分であり、ひずみテンソル$\varepsilon_{ij}$は、
変位を使って
\begin{equation}
	\varepsilon_{ij}=\frac{1}{2}(u_{i,j}+u_{j,i})
	\label{eqn:}
\end{equation}
で、応力テンソルは弾性係数テンソル$C_{ijkl}$として
を用いて次のように表される。
\begin{equation}
	\varepsilon_{ij}=C_{ijkl}\varepsilon_{kl}
	\label{eqn:}
\end{equation}
ただし、ここでは媒体は等方性体とみなし、
\begin{equation}
	C_{ijkl}=\lambda \delta_{ij}\delta_{kl} +\mu \delta_{ik}\delta_{jl}
	\label{eqn:}
\end{equation}
とする。ここに、$\lambda, \mu$はラメ定数を表す。ラメ定数と密度より、
縦波と横波の位相速度は、順に
\begin{equation}
	c_{P}=\sqrt{\frac{\lambda + 2\mu}{\rho}}
	, \ \ 
	c_{S}=\sqrt{\frac{\mu}{\rho}}
	\label{eqn:}
\end{equation}
と与えられる.工学でしばしば用いられるヤング率$E$,ポアソン比$\nu$,せん断剛性$G$を
用いれば、ラメ定数は
\begin{equation}
	G=\mu, 
	\label{eqn:}
\end{equation}
$\mu$はせん断合成に一致している。
以上の式を与えられた初期条件、境界条件の元で解くことで、超音波の伝播シミュレーションを行うことができる。
\section{数値解析法(FDTD法)}
