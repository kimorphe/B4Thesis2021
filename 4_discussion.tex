%\chapter{継手ディテールを考慮した超音波伝播シミュレーション}
この章では、T継手の詳細な形状を考慮したモデルを用いた超音波伝播解析を行う.
解析結果は、前章で示したより基本的なモデルでの結果をベースとして解釈する
とともに、超音波探傷試験の観点から見た意味について議論する。
ここでも、解析結果は、き裂の無いモデルで計算した入射波動場、
き裂があるモデルでの全波動場、両者の差として得た散乱波動場の順に示す。
その後、走時波形を示したのち、走時波形から検出できる散乱波成分が
どのようにして生じたものか、その超音波探傷試験における有用性について
議論を行う。
\section{解析モデル}
解析モデルの形状を図\ref{fig:4_0}に示す。
このモデルは、鋼床版Uリブの隅肉溶接部に生じた疲労き裂の超音波探傷を
想定したものである。フランジはデッキプレートに、ウェブはUリブの一部
を模擬したもので、ウェブはフランジに対して15度直交方向から
傾いている。き裂はルートギャップからデッキプレート側へ進展した場合の
モデルとするため、ウェブ左側脚部に1mm角のギャップを設け、
その端部からき裂を設けてある。また、ウェブ右側は隅肉溶接の余盛り
を表現するため、脚長7mmの余盛りとなるよう、曲率半径20mmの円弧で
囲われた領域を設けてある。
入射波の励起は、水平位置$x=5$mm, 幅1mmの区間に加えた一様な
鉛直荷重で、これまでの計算と同じ条件で行う。
以下、これら継手ディテールが、超音波伝播にどのような影響を与えるかという
点を中心にして、解析結果を検討する。
\begin{figure}[h]
	\begin{center}
	\includegraphics[width=0.7\linewidth]{Figs/model_bead.eps} 
	\end{center}
	\caption{
		隅肉溶接の余盛りを考慮した詳細なT継手モデル
	} 
	\label{fig:fig4_0}
\end{figure}
%--------------------
\begin{figure}[h]
	\begin{center}
	\includegraphics[width=1.0\linewidth]{Figs/bead_inc.eps} 
	\end{center}
	\caption{
		速度場$|\fat{v}(\fat{x},t)|$のスナップショット.
		P,Sは縦波,横波を,Rは表面波,Hはヘッド波を表す.
		ハイフンで繋がれた文字は反射前後でのモードを表す. き裂が存在しない場合.
	} 
	\label{fig:fig4_1}
\end{figure}
\subsection{入射波の挙動}
図\ref{fig:fig4_1}に、入射場の進展を速度場$|\fat{v}^{in}(\fat{x},t)|$の
スナップショットとし示す。入射波の様子は(a)の時点では、T継手の場合と大差なく、
P波が止端部で回折を起こした後、溶接部に進む。(a)は、表面波が止端部に
達した時点の様子を、(b)では表面波に起因した回折波が生じて余盛り表面に沿う
方向へ表面波の一部が進行している様子を示す。さらに時間が経過すると、(c)では
P-P波が余盛り部に侵入し、(d)では余盛り表面で強い反射波P-P-PとP-P-Sが生じる.
縦波反射波であるP-P-Pは、ウェブに達した後(e)にあるようにウェブ左側面へ、
横波P-P-Sはルートギャップに向かって進むことが(e)の図から分かる。
P-P-Sの波面は余盛りの形状を反映してレンズ状になっており、次第に収束する.
その結果(f)の時点では多重反射を繰り返す縦波と同程度の振幅を持つに至っている。
このような挙動は前章で用いた単純なT継手では現れず、余盛り形状の効果を
反映したものであることが明らかである。
\begin{figure}[h]
	\begin{center}
	\includegraphics[width=1.0\linewidth]{Figs/bead_tot.eps} 
	\end{center}
	\caption{
		速度場$|\fat{v}(\fat{x},t)|$のスナップショット.
		P,Sは縦波,横波を,Rは表面波,Hはヘッド波を表す.
		き裂がある場合.
	} 
	\label{fig:fig4_2}
\end{figure}
\subsection{き裂が存在する場合の波動場}
き裂を含む場合の全波動場の様子を\ref{fig:fig4_2}に示す。
ここでも、スナップショットとして表示する時刻は、代表的な入射波が
き裂に達する前後となることを基準に選んである。
(a)では、最初のP波がき裂に入射した直後で、前章のT継手モデルとほとんど同じ状況が
再現されている。(b),(c)に関しても概ね同様で、ウェブがフランジに対して傾いては居るが、
P-P波や、R-S波が継手内部を波面の形状を大きく乱すことなく進んでいる。
(d)からはレンズ状のP-P-Sの発生がこれまでとことなるが、R-S波は(d)から(e)の時点で
き裂に到達して後方散乱波を発生させている。(f)では、収束して大きな振幅をもった
P-P-Sがき裂近傍に到達している。
\begin{figure}[h]
	\begin{center}
	\includegraphics[width=1.0\linewidth]{Figs/bead_sct.eps} 
	\end{center}
	\caption{
		速度場$|\fat{v}(\fat{x},t)|$のスナップショット.
		P,Sは縦波,横波を,Rは表面波,Hはヘッド波を表す.
		き裂がある場合の散乱波成分のみを表示.
	} 
	\label{fig:fig4_3}
\end{figure}
\begin{figure}[h]
	\begin{center}
	\includegraphics[width=1.0\linewidth]{Figs/bead_sct2.eps} 
	\end{center}
	\caption{
		速度場$|\fat{v}(\fat{x},t)|$のスナップショット.
		P,Sは縦波,横波を,Rは表面波,Hはヘッド波を表す.
		き裂がある場合の散乱波成分のみを表示.
	} 
	\label{fig:fig4_4}
\end{figure}
\subsection{散乱波の発生と伝播}
き裂を含むモデルでの散乱波成分を抽出した結果を図\ref{fig:fig4_3}に示す。
ここに示した結果の内、(a)から(d)は、継手ディテールを考慮しない
前章のモデルで得られたものとほとんど同じであることが分かる。
一方(e)と(f)の時刻では、P-P波からき裂での散乱によってS波に転じたS2波が、
余盛りに向かって進行し、直接ウェブに入らない点がこれまでと異なっている。
また、継手を横断したR-S波に起因するS3後方散乱波は、簡易T継手モデルの場合
よりもやや振幅が小さくなっている。
この後に示す走時波形には、これらS2、S3波が大きな振幅をもって観測される
と予想されるため、ここでは、図\ref{fig:fig4_3}(f)の後の状況についても
見ておく。

図\ref{fig:fig4_4}は、図\ref{fig:4_3}に続く散乱波動場の様子を
0.5$\mu$sおきに示したものである。強いS3後方散乱波は、スピードが遅く
ゆっくりと右方向に波面と振幅をほとんど変化させずに進んでいる。
余盛り表面を経由してフォーカスのかかったS2波は、(a)の時間に
き裂上端部にあたり、(b)ではその結果モード変換した散乱波S2-Pが現れている。
この縦波は、(c)の図では右下に向かってき裂から離れていく様子が見られる。
その後ろには、同時に発生した横波散乱波S2−SがS2−Pと分離して現れる。
この横波散乱波S2-Sは大きな振幅を持ち、S2-Pよりも狭い範囲に集中したまま
進んでいく。これら,S2-P,S2-Sは、いずれもフランジ底面で反射して、
各々フランジ上面に向けて進むP波とS波を生じさせる。
これらは、余盛り表面が音響レンズとして作用した結果として生じたもので、
余盛り形状が既知であれば、非破壊検査に利用できる可能性のある成分としての
重要性がある。

\subsection{走時曲線}
最後に、走時曲線を図\ref{fig:fig4_5}に示す。
入射波、全波動場、散乱波を上から順に示すことは、これまでと同様である。
ここでも、入射波とき裂が存在する場合の全波動場の走時曲線は、単純なT継手
の場合と大差ない。一方で散乱波成分を抽出してみると、表面波(S3-R)が
非常に大きな振幅をもって伝わることはこれまで通りだが、
余盛りのレンズ効果に起因した散乱波がこのモデルの場合にのみ現れている。
このように、散乱波の走時曲線に現れるということは、入射波成分の分離という
課題は残るにせよ、実際に観測可能性があることを示唆し、観測が実現すれば
超音波探傷にも利用できると考えられる.
\begin{figure}[h]
	\begin{center}
	\includegraphics[width=0.8\linewidth]{Figs/bead_bwvs.eps} 
	\end{center}
	\caption{
		$x=0\sim35, y=12$mmの位置で得られた波形の走時プロット.
		(a)き裂なし,(b)き裂ありの場合.
		(c)はき裂ありと無しの場合の差分をとって計算した散乱波成分.
	} 
	\label{fig:fig4_5}
\end{figure}
%--------------------


