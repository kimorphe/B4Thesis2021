\chapter{超音波探傷の観点からの考察}
\begin{figure}[h]
	\begin{center}
	\includegraphics[width=1.0\linewidth]{Figs/bead_inc.eps} 
	\end{center}
	\caption{
		速度場$|\fat{v}(\fat{x},t)|$のスナップショット.
		P,Sは縦波,横波を,Rは表面波,Hはヘッド波を表す.
		ハイフンで繋がれた文字は反射前後でのモードを表す. き裂が存在しない場合.
	} 
	\label{fig:fig4_1}
\end{figure}
\begin{figure}[h]
	\begin{center}
	\includegraphics[width=1.0\linewidth]{Figs/bead_tot.eps} 
	\end{center}
	\caption{
		速度場$|\fat{v}(\fat{x},t)|$のスナップショット.
		P,Sは縦波,横波を,Rは表面波,Hはヘッド波を表す.
		き裂がある場合.
	} 
	\label{fig:fig4_2}
\end{figure}
\begin{figure}[h]
	\begin{center}
	\includegraphics[width=1.0\linewidth]{Figs/bead_sct.eps} 
	\end{center}
	\caption{
		速度場$|\fat{v}(\fat{x},t)|$のスナップショット.
		P,Sは縦波,横波を,Rは表面波,Hはヘッド波を表す.
		き裂がある場合の散乱波成分のみを表示.
	} 
	\label{fig:fig4_3}
\end{figure}
\begin{figure}[h]
	\begin{center}
	\includegraphics[width=1.0\linewidth]{Figs/bead_bwvs.eps} 
	\end{center}
	\caption{
		$x=0\sim35, y=12$mmの位置で得られた波形の走時プロット.
		(a)き裂なし,(b)き裂ありの場合.
		(c)はき裂ありと無しの場合の差分をとって計算した散乱波成分.
	} 
	\label{fig:fig4_4}
\end{figure}
%--------------------


