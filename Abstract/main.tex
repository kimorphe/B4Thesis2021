%%#!platex
%
% Example of Japanese Paper of JSCE
% for LaTeX2e users
%
% revised on 4/25/2014
%
%%%%%%%%%%%%%%%%%%%%%%%%%%%%%%%%%%%%%%%%%%%
%
% もし jis フォントメトリックを使う場合は,以下をアンコメントしてください.
% \DeclareFontShape{JY1}{mc}{m}{n}{<-> s * jis}{}
% \DeclareFontShape{JY1}{gt}{m}{n}{<-> s * jisg}{}
%
\documentclass{jsce}
%
\usepackage{epic,eepic,eepicsup}
%\usepackage{graphicx,multicol}
\usepackage{graphicx}
\usepackage{multicol}
\usepackage{amsmath}
%\usepackage{showkeys}
\usepackage{setspace}
%  amsを使う方は以下をアンコメントしてください.
%\usepackage{amssymb,amsmath}
% 英語はサポートしているかどうか不明
% \inenglish
% 学会サンプルに times とあるので指定しておきます
\usepackage{times}
%
\finalversion
%\pagestyle{empty}
%
\title{
T字溶接継手部のき裂による\\
超音波エコーの励起・伝搬メカニズム解析
}%
\endtitle{
STUDY ON CHEMICAL IMPEDANCE CHARACTERISTICS OF UNSATURATED COMPACTED CLAY
}
%
% emailアドレスのフォントをタイプライター体にしたい方は次行をアンコメント
% \emailstyle{\ttfamily}
% emailアドレスを公開される方は,
%% \thanks{○○○○○○\email{your_name@foo.ac.jp}}のようにしてください.
%
\author{
 10430231  永井 瑞希
\thanks{岡山大学環境理工学部・環境デザイン工学科 (〒700-8530 岡山県岡山市北区3丁目1番地1号)}
}
\endauthor{Mizuki NAGAI}
%
\abstract{
\small
本研究では,溶接部の超音波探傷試験の信頼性を担保する上で有用な知見を得ることを目的に,T継手のき裂による超音波エコーの励起・伝播メカニズムを,2次元数値シミュレーションで調べた.シミュレーションには時間領域差分法(FDTD法)を用い,溶接ルートから進展したき裂を模擬した解析モデルで,散乱波の発生と伝播挙動を検討した.
その際,全てのシミュレーションをき裂の有無だけを変えたモデルで行い,両者の差から散乱波成分を抽出した.また,各種散乱波の成因を明らかにするために,入射波がき裂に到達する時間を超音波モード毎に調べ,散乱波の発生タイミングと入射波およびモードの対応を明らかにした.さらに,散乱波による供試体表面の鉛直動を走時波形として示し,超音波探傷試験で観測しうるエコー成分と,き裂から観測点への伝搬経路を明らかにした.以上の結果,大きな振幅を有する表面波と,到達時間が早く複数経路から到達する縦波,それぞれの後方散乱波がき裂検出に有効であることが分かった.さらに,余盛り表面がレンズ効果を果たすことで,強い横波散乱波が発生すること現象を見出し,このエコーを用いることで新たな探傷法開発につながるとの知見を得た.これらの検討は,今後,T継手のき裂探傷条件を最適化する上で有用な情報となる.

}
%
\keywords{ultrasonic flaw detection, numerical simulation, finite difference time-domain, welding joint, scattering}
%
\endabstract{% Yes blank line
\normalsize
This study investigates the propagation characteristics of surface wave traveling in a random heterogeneous medium. For this purpose, ultrasonic measurements are performed on a coarse-grained granite block as a typical randomly heterogeneous medium. In the ultrasonic testing, a line-focus transducer is used to excite ultrasonic waves, whereas a laser Doppler vibrometer is used to pickup the ultrasonic motion on the surface of the granite block. The measured waveforms are analysed in the frequency domain to evaluate the travel-time for each measurement point based on the Fermat's principle. From the ensemble of travel-times obtained thus, 
the probability distribution of the travel-time is established as a function of travel-distance. The uncertainty of the travel-time and its spatial evolution are then investigated using the standard deviation of the travel-time as a measure of the uncertainty. As a result, it was found that the uncertainty is 
approximately proportional to the mean travel-time divided by the square root travel-distance.This is a finding that would be useful for stochastic modeling of the waves in random heterogeneous media. 
}
%
% \titlepagecontrol{1}
%
%\receivedate{2019.7.19}
% \receivedate{January 15, 1991}
%
% \def\theenumi{\alph{enumi}}  % もし enumerate 最初の箇条を (a) と
% \def\labelenumi{(\theenumi)} % したい場合・・・
%
\begin{document}
\maketitle
%%%%%%%%%%%%%%%%%%%%%%%%%%%%%%%%%%%%%%%%%%%%%%%%%%%%%%%%%%%%%%%%%%%%%%%%%%%%%
\section{はじめに}
鋼橋の維持管理において,疲労き裂を検出するために超音波探傷試験が行われている.超音波探傷試験は,表面近傍だけでなく内部き裂の検出や評価が可能という点で磁粉探傷や渦電流法等の非破壊検査法に対する優位性をもつ.弾性波の一種である超音波は,部材やきず表面など,物性が変化する箇所で反射や散乱エコーを発する.超音波探傷法はこのことを利用し,部材表面で観測したエコーからき裂や表面の存在を検知する.超音波の伝播速度が既知であれば,エコー到達時刻から伝播距離が見積もられ,きずや表面の位置を推定することができる.超音波探傷法の原理はごく簡単なものである.一方で,溶接継手部のき裂検出や評価を精確かつ高い信頼性をもって行うことは必ずしも簡単ではない.その理由の多くは,超音波伝播特性の複雑さに起因する.超音波は部材表面では反射するだけでなく,モード変換によって縦波や横波に分離する.そのため,検査領域へ送信された超音波は部材表面で反射するたび二つのモードに分岐し,伝播経路を多数かつ複雑にする.さらに,超音波エコーの発生源は,検出すべきき裂だけでなく,部材端部や角部,溶接止端部などもエコーを発する.このような超音波の性質上,超音波エコーがどのような経路を経たものか,それが目的とするき裂に由来するものかを正しく判断することは単純な作業でなく,比較的単純な形状をしたT継手の探傷も例外でない.
T継手は二つの板材が直角に近い向きで接合されている.各々の板内部は超音波が多重反射され,溶接部には止端やルートなど,き裂以外の散乱源もある.その結果,き裂と超音波送受信点の配置によっては,き裂エコーが観測出来ないケースや,周辺境界からのエコー(形状エコー)と区別し難しい状況が生じる.さらに,溶接ビードのサイズや形状は検査箇所によって完全には同じでないため,き裂エコーと形状エコーを単純な規則で機械的に区別することは困難である.このことから,溶接継手における超音波探傷を十分な信頼性をもって行うには,継手内部と周辺で散乱波がどのように励起,伝搬するか十分に理解することが重要となる.

本研究ではこのことを踏まえ,溶接継手における超音波伝播挙動の把握を目的とした数値シミュレーションを行う.特に,T継手角部から進展したき裂の超音波探傷を模擬し,き裂エコーの励起と伝搬挙動を詳しく調べる.これは,送受信条件の制約が厳しく,探傷が難しいとされる鋼床版Uリブ継手の検査を念頭においたものである.超音波探傷試験のシミュレーションは,これまでにも有限要素法などを用いて行われている.数値解析では,波動場の挙動を任意の位置と時刻で調べられることが利点である.ただし,き裂エコーは入射波や形状エコーに比べて微弱で,数値解析結果でも発生や伝播状況が自動的にわかり易く表示される訳ではない.従って,数値解析結果を
弾性波理論に基づき正しく解釈する工夫が必要になる.この点に関して本研究は次のことを行う.
\begin{itemize}
\item
	き裂の有無だけが異なる2つのモデルのペアで解析を行い,両者の差として散乱波成分を正確に抽出する.
\item
	き裂への入射波の到達時刻をモード毎に調べ,散乱波の発生タイミングと対応する入射波モードを特定する.
\item
	超音波探傷波形をシミュレーション結果から求め,観測しうるき裂エコー(散乱波)を特定してその成因を調べる.
\end{itemize}
以下では,数値シミュレーションで想定する探傷条件と数値解析法について述べる.続いて,代表的な数値シミュレーション結果を示し,入射波の進展と散乱波の発生状況を調べる.最後に,超音波探傷に利用可能と考えられるエコーを特定し,その成因や影響因子について議論する.
\section{数値シミュレーション手法の概要}
\subsection{問題設定}
図\ref{fig:fig1}に,本研究で想定するT継手の超音探傷試験の状況を示す.
(a)は鋼床版Uリブの模式図を示し,丸で囲った箇所が探傷すべき継手部にあたる.
ただし,溶接ビードや余盛りの形状はここには示していない.
この部分を上下反転して拡大したのが(b)の図である.
以下では,便宜上,床版デッキプレート側をフランジ,Uリブをウェブと呼ぶ.
Uリブは閉断面を構成するので,このT継手の左側を内,右側を外と呼ぶ.
探傷は継手外側,デッキプレート下からのみ可能なため,図\ref{fig:fig1}(b)のフランジ上面を探傷面(送受信可能な範囲)と設定する.き裂は,継手内側のコーナー部分からフランジに貫入するものを考える.
き裂に向け,フランジ上面の探傷面に設置した探触子(超音波センサー)から入射波を送信し,青線で示した範囲の各点でエコー観測を行う.受信センサーは示していないが,このような観測は,十分な素子数をもつアレイセンサーとマルチプレクサを使って実現できる.なお,受信センサーは,設置面に対する鉛直動を取得するものとする.
%--------------------
\begin{figure}[h]
	\begin{center}
	\includegraphics[width=1.0\linewidth]{Figs/fig1.eps} 
	\end{center}
	\caption{
		鋼床版Uリブ接合部(a)をイメージしたT継手の超音波探傷試験(b).
	} 
	\label{fig:fig1}
\end{figure}
%--------------------
\subsection{数値解析法}
数値解析には時間領域差分法(FDTD法)を用いる.解析を多数のケースで実施するため,ここでは計算コストの低い2次元問題として問題を扱う.FDTD法は有限差分法の一種で,速度と応力を未知量として,時間,空間とも中央差分で離散化する.時間ステッピングはリープフロッグ法により陽解法で行う.FDTD法は,実装が簡単で計算効率が高いことが知られている.ただし,陽解法のため時空間の離散化幅は,クーランの安定条件を満足するように設定する必要がある.空間格子は等間隔の正方格子系を用い,入射波波長に対して十分な数の離散化格子を取る.なお,計算量を抑えるため,物理的な境界を打ち切って計算領域を定める.その際,人為的に設定された計算領域の境界を吸収領域につなぎ,反射が生じないようにする.吸収領域にはよく知られたPML(perfectly matched layer)を用いる.
\subsection{解析モデル}
図\ref{fig:fig2}に差分法解析に用いたT溶接継手モデルを示す.
これは,ウェブとフランジが隅肉溶接で接合されたケースを想定したものである.
き裂は,ルートギャップ端部からフランジへ伸び,向きは鉛直,深さ4mmとした.
また,溶接の余盛りは脚長7mm,外形は半径20mmの円弧で与えた.
入射超音波は,フランジ上面に設置した探触子で行う.
これを模擬するため,入射波を鉛直荷重で励起する.
その位置は溶接止端から5mm,鉛直力の分布幅は1mmとした.
鉛直力の時間変化は,周波数を5MHzのコサイン波をガウス分布で振幅変調したパルス波で与える.
このようにして励起された入射波に対して発生する継手内部の波動場を可視化し,き裂散乱波の挙動を調べる.
なお,探傷結果として得られるものは,探傷面上での振動波形であることから,図に観測範囲として示した$0\leq x\leq 45$mm, $y=12$mmで得られる鉛直動$v_2$の波形を観測波形(探傷波形)とみなす.
差分法での離散化は,空間格子間隔を$\Delta x=$0.05mm, 時間ステップ間隔を$\Delta t=$0.003$\mu$sとし,その結果,PML領域を含めて格子数は合計1,600$\times$800=1,280,000点,時間ステップ数は20,000となった.
%--------------------
\begin{figure}[h]
	\begin{center}
	\includegraphics[width=1.0\linewidth]{Figs/fig2.eps} 
	\end{center}
	\caption{
		T継手における超音波探傷試験の差分法解析モデル.
	} 
	\label{fig:fig2}
\end{figure}
%--------------------
\section{シミュレーション結果}
\subsection{き裂を持たないモデルに対する結果(入射波の挙動)}
はじめに,き裂を持たないモデルで得た入射波の伝播挙動を図\ref{fig:fig3}に示す.
この計算は,き裂を持たないことを除き,条件は図\ref{fig:fig2}に示すものと同じである.
図\ref{fig:fig3}は,6つの時刻における速度場のスナップショットをカラーマップ表示したものである.
白の実線は継手の輪郭を示し,(a)の図には鉛直力を加えた箇所を示してある.
なお,速度ベクトルの振幅値は,鉛直力と密度を1とおいたときの相対値である.
暖色系の明るく示された箇所は強い振動が起きている部分で,半円や半楕円に近い形をしている.
これらひと続きの円や楕円状の軌跡は,異なる経路で伝わった弾性波モードの波面に対応する.
そのため図\ref{fig:fig3}では,主要な入射波モードにはP,SおよびRを組み合わせたラベルをつけてある.
これらは順にP波(縦波),S波(横波),レイリー波(表面波)を表す.
例えば,(a)の図にあるラベルPは,入射点から最も早く伝搬するP波の波面(到達位置)を示す.
また,ラベルP-Pは,P波として入射した波がフランジ底面で反射した後,再びP波として伝わったものであることを表す.同様にP-Sは,P波がモード変換で転じたS波を意味する.
このように,それぞれの波面は,時間経過による位置の変化を追跡することで,対応するモードとモード変換の履歴を記載することができる.より詳しい説明は紙面の都合上本論文に譲り,ここでは,後の議論に関係する重要な点だけを説明する.\\
図\ref{fig:fig3}(b)のRは,右方向に進展する表面波である.
表面波は,入射直後は左右両方へ同じスピードで伝わる.
このうち左側に進んだ表面波は,溶接ビードに到達した時点で回折を起こし,余盛り表面を這う表面波と余盛り内部に浸入するS波に分離する.後者の波面は図の(a)でR-Sと記したものである.R-S波はそのまま継手内部を進み,き裂の起点であるルートギャップに向かう.R-S波は後でみるように強い横波散乱波を励起する.
次に注目すべきモードは,(c)の時刻で発生するP-P-S波である.
これは,フランジ底面からの反射波P-Pが,ウェブに向かって進行し,余盛り表面で反射されて生じたS波である.
このことから,P-P-Sのラベル付している.
余盛り表面での反射はP波,すなわちP-P-Pも生じるが,特筆すべきはP-P-Sの挙動である.
P-P-S波は,(d)の図では余盛り表面から離れレンズ状の波面としてルートギャップに向かっている.
これは凸な余盛りの形状の効果で,更に時間経過した(e)ではP-P-S波が狭い範囲に集束して大きな振幅を持つようになる.最終(f)の時点ではこの傾向が一層進んでルートギャップ近傍に達する.
これは,凸な余盛り形状が音響レンズのように作用して横波を集束させたもので,き裂のあるモデルでは,強い散乱波を発生させる.レンズ効果の程度は余盛り形状で異なると予想され,もし凹な形状であれば,逆のレンズ効果でP-P-Sは拡散する.一方,凸な余盛りであればある程度のレンズ効果が見込まれ,その意味では普遍性のある現象と考えることができる.
\begin{figure}[h]
	\begin{center}
	\includegraphics[width=1.0\linewidth]{Figs/fig3.eps} 
	\end{center}
	\caption{
		入射波の進展挙動(き裂を持たないモデルでの計算結果).
	} 
	\label{fig:fig3}
\end{figure}
\subsection{き裂を有するモデルに対する結果(散乱波の挙動)}
次に,図\ref{fig:fig3}の入射波に対して発生する散乱波の挙動を見る.
図\ref{fig:fig5}は,き裂を有するモデル(\ref{fig:fig2})での結果から,入射波を差し引いた散乱波成分だけを示したものである.スナップショットを示した6つの時刻は,図\ref{fig:fig3}と同じで,二つの図を対照することで,いつ,どの入射波成分によって散乱波が生じたか調べることができる.
例えば(a)でP1と書いた波面は縦波散乱波で,図\ref{fig:fig3}(a)より,き裂右側からP波が到来した直後
に発生したものであることが分かる.すなわち,送信点から直接到来したP波に励起された,縦波後方散乱波であることが理解できる.逆に,図\ref{fig:fig3}(d)では,溶接ビード内部を進んだR-S波が,丁度,ウェブを通過して,本来き裂がある箇所に達しようとしている.そこで,散乱波の分布を図\ref{fig:fig5}の(d)と(e)で見ると,(d)では存在しない横波S3が(e)の時刻に発生している.これがR-S波で励起された散乱波であることは明らかで,この後,観測領域に向かって大きな振幅を持ったまま伝わることが確認できる.このように,入射波と散乱波の到達および発生タイミングを見ることで,いつどのようなメカニズムで散乱波が生じたかを理解できる調べることができる.
なお,前項で述べた,レンズ効果を受けて集束するP-P-S波は,(f)の時刻直後にP波とS波両方の散乱波を
右下方向に放射する.これが,フランジを伝わって超音波エコーとして観測される.この様子は紙面の制約上,本稿では割愛する.
%\begin{figure}[h]
%	\begin{center}
%	\includegraphics[width=1.0\linewidth]{Figs/fig4.eps} 
%	\end{center}
%	\caption{
%		き裂が存在するモデルでの解析結果
%	} 
%	\label{fig:fig4}
%\end{figure}
\begin{figure}[h]
	\begin{center}
	\includegraphics[width=1.0\linewidth]{Figs/fig5.eps} 
	\end{center}
	\caption{
		散乱波の発生と伝搬状況.
	} 
	\label{fig:fig5}
\end{figure}
\subsection{模擬探傷波形}
最後に,$0\leq x\leq 45$mm,$y$=12mmの観測範囲で得られた,フランジ表面の鉛直動分布を図\ref{fig:fig6}に示す.
この図は,横軸を時間,縦軸を位置($x$)にとり,各時空間点における鉛直速度を示した走時波形である.
3つの走時波形は上から順に(a)き裂を持たないモデルで計算した入射波成分,(b)き裂を有するモデルで得られた全速度成分,および(c)二つの走時波形(b)と(c)の差として得られた散乱波成分を示す.
超音波探傷では(b)のような走時波形が実際に計測され,
図\ref{fig:fig4}や図\ref{fig:fig5}にのような部材内部の状況を見ることはできない.
また,入射波成分だけを実験で得ることは,き裂の無い正確なレプリカを作成して,正確に同じ位置で計測することが必要で通常実施が困難である.ここで,図\ref{fig:fig5}(c)の散乱波成分を見ると,10$\mu$s付近から右上がりの直線が現れている.これは,R-S波で励起された散乱波で,(b)の走時波形中でも容易に認識できる強度がある.一方,18$\mu$付近から伸びる曲線は,余盛りのレンズ効果で集束されて散乱した横波であることが,図\ref{fig:fig5}以後の時刻で散乱波と入射波の空間分布を照合することで明らかにできる.
これ以外に,成因の特定できるものは,早い時間に到達す縦波後方散乱波P1がある.
P1は,他の波と時間的に近接することなく現れるという点で探傷上有利である.一方でP1の鉛直振動は他の散乱波成分に比較して小さいことから,実際に計測するためには,縦波振動の方向に合わせた高感度な受信センサーを用いる等の工夫が必要になることが分かる.
\begin{figure}[h]
	\begin{center}
	\includegraphics[width=1.0\linewidth]{Figs/fig6.eps} 
	\end{center}
	\caption{
		数値シミュレーションで得た模擬超音波探傷波形.上から,入射波,全成分,および散乱波成分の走時波形.
	} 
	\label{fig:fig6}
\end{figure}
%--------------------
\section{まとめ}
本研究ではT字溶接継手におけるき裂の超音波探傷試験を模擬した数値シミュレーションを行い,き裂エコーの励起,伝搬挙動を詳しく調べるともに,実際に観測しうるエコーを模擬探傷波形において特定した.
その結果,大きな振幅を持つ表面波や,早い時間帯に到達する縦波散乱波の観測が探傷上有望であることが分かった.さらに,余盛り表面のレンズ効果を受けて集束される横波も,散乱波の探傷波形中に比較的明確に現れることを示した.なお,散乱波成分のみの探傷波形は実際には計測出来ないが,本研究で行ったように,数値シミュレーションによって発生位置と到達時刻を特定することがきれば,その測定に特価したセンサーの開発や送受条件の設定が可能となり,超音波探傷の高精度化や信頼性の向上に有用な情報を与える.今後は,実験と数値シミュレーションの比較を行い,シミュレーションの有効性を検証すること,き裂向きや大きさを変化させたときの探傷波形パターンの変化を調べ,実際の探傷に応用することが課題となる.
%%%%%%%%%%%%%%%%%%%%%%%%%%%%%%%%%%%%%%%%%%%%%%%%%%%%%%%%%%%%%%%%%%%%%%%%%%%%%
%{\gt 謝辞:}
%本実験に用いた花崗岩供試体は浮田石材店代表浮田隆司氏に提供頂いた.
%また本研究の推進には,科学研究費補助金(基盤研究©課題番号\#18K04334)の補助を受けた.
%併せて謝意を表す.
%%%%%%%%%%%%%%%%%%%%%%%%%%%%%%%%%%%%%%%%%%%%%%%%%%%%%%%%%%%%%%%%%%%%%%%%%%%%%
\end{document}
%\newpage
%\lastpagecontrol[2cm]{13.7cm}
\begin{thebibliography}{99}
\begin{spacing}{1.175}
\bibitem{Itagaki}
	板垣 昌幸:電気化学インピーダンス法 第二版 原理・測定・解析,丸善出版, 2011. 
\end{spacing}
\end{thebibliography}
%\begin{flushright}
%	\small
%	\bf{ (Received June 24, 2020)\\
%	(Accepted November 19, 2020)}
%\end{flushright}
\end{document}

%\lastpagesettings
%\begin{minipage}[c]{13.7cm}
%\end{minipage}
%\lastpagecontrol[0cm]{13.7cm}
%\begin{multicols}{1}
%-------------------------------------------------
%-------------------------------------------------
%\end{multicols}

