%%#!platex
%
% Example of Japanese Paper of JSCE
% for LaTeX2e users
%
% revised on 4/25/2014
%
%%%%%%%%%%%%%%%%%%%%%%%%%%%%%%%%%%%%%%%%%%%
%
% もし jis フォントメトリックを使う場合は,以下をアンコメントしてください.
% \DeclareFontShape{JY1}{mc}{m}{n}{<-> s * jis}{}
% \DeclareFontShape{JY1}{gt}{m}{n}{<-> s * jisg}{}
%
\documentclass{jsce}
%
\usepackage{epic,eepic,eepicsup}
%\usepackage{graphicx,multicol}
\usepackage{graphicx}
\usepackage{multicol}
\usepackage{amsmath}
%\usepackage{showkeys}
\usepackage{setspace}
%  amsを使う方は以下をアンコメントしてください.
%\usepackage{amssymb,amsmath}
% 英語はサポートしているかどうか不明
% \inenglish
% 学会サンプルに times とあるので指定しておきます
\usepackage{times}
%
\finalversion
%\pagestyle{empty}
%
\title{
T字溶接継手部のき裂による\\
超音波エコーの励起・伝搬メカニズム解析
}%
\endtitle{
STUDY ON CHEMICAL IMPEDANCE CHARACTERISTICS OF UNSATURATED COMPACTED CLAY
}
%
% emailアドレスのフォントをタイプライター体にしたい方は次行をアンコメント
% \emailstyle{\ttfamily}
% emailアドレスを公開される方は,
%% \thanks{○○○○○○\email{your_name@foo.ac.jp}}のようにしてください.
%
\author{
 10430231  永井 瑞希
\thanks{岡山大学環境理工学部・環境デザイン工学科 (〒700-8530 岡山県岡山市北区3丁目1番地1号)}
}
\endauthor{Kengo SASAKI}
%
\abstract{
\small
本研究では超音波探傷試験を高度化する上で有用な知見を得ることを目的に,T隅肉溶接継手のき裂による超音波エコーの
励起と伝播メカニズムを2次元数値シミュレーションで調べた.数値シミュレーションには時間領域差分法(FDTD法)を用い,
溶接ルートから進展したき裂を模擬した解析モデルで,散乱波の発生と伝播挙動を詳細に検討した.
その際,全てのシミュレーションをき裂を含むものと含まないものの両方で行い,両者の差から散乱波成分だけを抽出した.
また,各種散乱波の成因を明らかにするために,入射波がき裂に到達する時間をモード毎に調べ,
入射波経路と対応する散乱波の発生タイミングを明らかにした.さらに,供試体表面の散乱波による鉛直動を走時波形と
して示し,超音波探傷試験で観測しうるエコー成分とき裂からの伝播経路を明らかにした.
以上の結果,大きな振幅を有する表面波と,到達時間が早く複数の経路から到達する縦波それぞれの後方散乱波が
き裂検出に有効であることが分かった.さらに,継手の余盛表面がレンズの効果を果たすことで,強い横波散乱波が発生
することを見出し,このエコーを用いることで新たな探傷法開発につながる知見を得た.以上の検討は,今後,T継手の
き裂探傷条件を最適化する上で有用な情報を与える.

}
%
\keywords{ultrasonic flaw detection, numerical simulation, finite difference time-domain, welding joint, scattering}
%
\endabstract{% Yes blank line
\normalsize
This study investigates the propagation characteristics of surface wave traveling in a random heterogeneous medium. 
For this purpose, ultrasonic measurements are performed on a coarse-grained granite block as a typical randomly heterogeneous medium. In the ultrasonic testing, a line-focus transducer is used to excite ultrasonic waves, whereas a laser Doppler vibrometer is used to pickup the ultrasonic motion on the surface of the granite block. The measured waveforms are analysed in the frequency domain to evaluate the travel-time for each measurement point based on the Fermat's principle. From the ensemble of travel-times obtained thus, 
the probability distribution of the travel-time is established as a function of travel-distance. The uncertainty of the travel-time and its spatial evolution are then investigated using the standard deviation of the travel-time as a measure of the uncertainty. As a result, it was found that the uncertainty is 
approximately proportional to the mean travel-time divided by the square root travel-distance.This is a finding that would be useful for stochastic modeling of the waves in random heterogeneous media. 
}
%
% \titlepagecontrol{1}
%
%\receivedate{2019.7.19}
% \receivedate{January 15, 1991}
%
% \def\theenumi{\alph{enumi}}  % もし enumerate 最初の箇条を (a) と
% \def\labelenumi{(\theenumi)} % したい場合・・・
%
\begin{document}
\maketitle
%%%%%%%%%%%%%%%%%%%%%%%%%%%%%%%%%%%%%%%%%%%%%%%%%%%%%%%%%%%%%%%%%%%%%%%%%%%%%
\section{はじめに}
鋼橋の維持管理において,疲労き裂を検出するために超音波探傷試験が行われている.
超音波探傷試験は,表面近傍だけでなく内部き裂の検出や評価が可能であるという点で磁粉探傷法や
渦電流法等の非破壊検査法に対する優位性をもつ.弾性波の一種である超音波は,部材やきず表面など,
材料物性が変化する箇所で反射や散乱によってエコーを発する.超音波探傷法ではこのことを利用し,
部材表面で観測したエコーから逆にき裂や表面の存在を検知する.
超音波の伝播速度が既知であれば,エコー到達時刻から伝播距離が見積もられ,
きずや表面の位置を推定することができる.
超音波探傷法の原理はこのようにごく簡単なものである.一方で,溶接継手部のき裂検出や評価を
超音波探傷法で精確かつ高い信頼性をもって行うことは必ずしも簡単ではない.
その理由の多くは,超音波伝播特性の複雑さに起因する.
超音波は部材表面では反射するだけでなく,モード変換によって縦波や横波に分離する.
そのため,検査領域に向けて送信された超音波は部材表面で反射するたび二つのモードに
分岐し,伝播経路は多数かつ複雑なものになる.さらに,超音波エコーの発生源は検出すべき
き裂だけに限らず,部材端部や溶接止端部などの角部もエコー発生源となる.
このような超音波の性質上,超音波エコーがどのような経路を経たものか,
また,それが目的とするき裂に由来するものかを正しく判断することは簡単でなく,
比較的単純な形状の継手であるT継手における探傷も例外でない.
T継手は二つの板材が直角に近い向きで接合されている.
各々の板内部は超音波が多重反射され,溶接部には止端やルート部を始めき裂以外に
複数の散乱源がある.その結果,き裂と超音波送受信点の配置によっては,き裂エコーが
観測出来ないケースや,周辺境界からのエコー(形状エコー)と区別が難しい状況が頻繁に
生じる.さらに,溶接ビードのサイズや形状は検査箇所毎に完全には同じでないため,
き裂エコーと形状エコーを単純な規則に従って機械的に区別することは困難である.
このことから,溶接継手における超音波探傷を十分な信頼性をもって高精度に行う
ためには,継手内部と周辺で散乱波がどのように励起され,進展するかを十分に理解
することが重要となる.

本研究ではこのことを踏まえ,溶接継手における超音波伝播挙動の把握を目的とした
数値シミュレーションを行う.特に,T継手の角部から進展したき裂の超音波探傷を
数値シミュレーションで模擬し,き裂エコーの励起と伝播挙動を詳しく調べる.
これは,送受信条件の制約が厳しく,探傷が難しいとされる鋼床版Uリブ継手の検査
を念頭においたものである.超音波探傷試験の高度化を目的とした数値波動解析は
これまでにも有限要素法などを用いて行われている.そのような数値解析では,
波動場の挙動を任意の位置と時刻に対して調べられることが利点である.
しかしながら,き裂エコーは入射波や形状エコーに比べて微弱で,数値解析結果においても
発生や伝播状況が自動的にわかり易く表示される訳では無い.従って,数値解析結果を
弾性波理論に基づき正しく解釈する工夫が必要になる.この点に関して本研究は次の
ことを行う.
\begin{itemize}
\item
	き裂を含むモデルと含まないモデルで二重に解析を行い,両者の差分として
	散乱波成分を正確に抽出する.
\item
	き裂への入射波の到達時刻を波動モード毎に調べ,散乱波の発生タイミングと
	対応する入射波モードを特定する.
\item
	超音波探傷波形をシミュレーション結果から合成し,観測しうるき裂エコー
	(散乱波)を特定してその成因を調べる.
\end{itemize}
以上を通じて,T継手内のき裂による散乱波のなかで,超音波探傷に利用可能な
エコーを判別し,その励起と伝播挙動を明らかにする.
以下では,数値シミュレーションで想定する探傷条件と,数値解析法について述べ
た後,シミュレーション結果を示す.続いて,代表的な数値シミュレーション結果を
示し.入射波の進展と散乱波の発生状況を調べる.最後に,超音波探傷に利用可能と
考えられるエコーを特定し,その成因や影響因子について議論する.
\section{数値シミュレーション手法の概要}
\subsection{問題設定}
図\ref{fig:fig1}に,本研究で想定するT継手の超音探傷試験の状況を示す.
(a)は鋼床版Uリブの模式図を示し,丸で囲った箇所が探傷すべき継手部にあたる.
ただし,溶接ビードや余盛りの形状は示していない.
この部分を上下反転して拡大したのが(b)の図で,T継手とみなすことがきる.
以下では,便宜上,床版側をフランジ,リブ側をウェブと呼ぶ.
Uリブは閉断面を構成しT継手の左側を内,右側を外と呼ぶ.
探傷は継手外側からのみ可能なため,図\ref{fig:fig1}(b)にあるように,
フランジ上面を探傷面(送受信可能な範囲)と設定する.き裂は,継手内側の
コーナー部分からフランジに貫入するものを考える.
フランジ上面の探傷面に設置した探触子(超音波センサー)から入射波を送信し,
青線で示した範囲各点でエコー観測を行う.
この図には受信センサーは示していないが,このうな観測は,
十分な素子数をもつリニアアレイセンサーとマルチプレクサを使って実現できる.
なお,受信センサーは,設置位置で上下動を取得するものとする.
%--------------------
\begin{figure}[h]
	\begin{center}
	\includegraphics[width=1.0\linewidth]{Figs/fig1.eps} 
	\end{center}
	\caption{
		鋼床版Uリブ接合部をイメージしたT継手の超音波探傷試験
	} 
	\label{fig:fig1}
\end{figure}
%--------------------
\subsection{数値解析法}
数値解析には時間領域差分法(FDTD法)を用いる.数値解析は,多数のケースで
実施する必要があるため,計算コストの低い2次元問題として行う.
FDTD法は有限差分法の一種で,速度と応力を未知量として,時間,空間とも
中央差分で離散する.時間ステッピングはリープフロッグ法で陽解法として
行う,実装が簡単で計算効率が高いことが知られている.ただし,時間と空間の
離散化幅は,クーランの安定条件を満足するように設定する必要がある.また,
空間的な格子配置は等間隔の正方格子系を用いるため,媒体の境界は階段状
にしか近似できない.しかしながら,入射波の波長に対して十分な数の
離散化格子を取れば,滑らかに変化する境界付近の波動場も十分な精度で
得ることができる.なお,計算量を抑えるため,物理的な境界を打ち切って
計算領域を定める.その際には,人為的に設定された計算領域の境界を
吸収領域につなぎ,反射はが生じないようにする.吸収領域にはよく
知られたPML(perfectly matched layer)を用いる.
\subsection{解析モデル}
図\ref{fig:fig2}に差分法解析に用いたT溶接継手のモデルを示す.
これは,T継手周辺を切り出したモデルで,ウェブとフランジは隅肉溶接で
接合されたケースを想定したものである.き裂は,ルートギャップ端部から
フランジに伸びるものとし,向きは鉛直,深さは4mmとした.
また,溶接余盛りは脚長7mmで,外形は半径20mmの円弧で与えた.
入射超音波は,フランジ上面に設置した探触子を模擬した,鉛直荷重で
励起する.その位置は溶接止端から5mm,鉛直力の分布幅は1mmとした.
これはアレイ探触子1素子を模擬したもので,時間的には,周波数を5MHz
のコサイン波をガウス分布で振幅変調したパルス波を加える.
このようにして励起された入射波に対して発生する継手内部の波動場を
計算および可視化し,き裂散乱波の挙動を調べる.なお,探傷結果として
得られるものは,探傷面上での振動波形であることから,
図に観測範囲として示した$0\leq x\leq 45$mm, $y=12$mmにおいて
得られる鉛直動の波形を観測波形とみなす.
差分法での離散化,空間格子間隔を$\Delta x=$0.05mm, 時間ステップの間隔を
$\Delta t=$0.003$\mu$sとした.その結果空間格子数はPML領域を含めて
合計1,600$\times$800=1,280,000点,時間ステップ数は20,000とした.
%--------------------
\begin{figure}[h]
	\begin{center}
	\includegraphics[width=1.0\linewidth]{Figs/fig2.eps} 
	\end{center}
	\caption{
		T継手における超音波探傷試験の差分法解析モデル.
	} 
	\label{fig:fig2}
\end{figure}
%--------------------
\section{シミュレーション結果}
\subsection{き裂の無いモデルに対する結果(入射波の挙動)}
はじめに,き裂が無いモデルを使った計算によって得た,入射波の伝播挙動を図\ref{fig:fig3}に示す.
この計算は,き裂が無いことを以外には,全て\ref{fig:fig2}と同じ条件で行った計算の結果である.
図\ref{fig:fig3}では,6つの時刻における速度場の分布(スナップショット)をカラーマップ
として示したものである.白の実線は継手の輪郭を示し,(a)の図には鉛直力を加えた箇所を書き加えている.
なお,速度ベクトルの振幅値を表すカラーバーの値は,鉛直力と密度を1とおいたときの相対値である.
暖色系の明るく示された箇所が,強い振動が起きている部分で,多くの場合半円や半楕円に近い形を
している.これらひと続きの円や楕円状の軌跡は,異なる経路で伝わった弾性波モードの波面に対応する.
図\ref{fig:fig3}では,主要な弾性波モードにはP,SおよびRの組み合わせのラベルをつけてある.
これらは順にP波(縦波),S波(横波),レイリー波(表面波)を意味する.
例えば,(a)の図にあるラベルPは,最も早く入射点から進行するP波の波面(到達位置)を示す.
また,ラベルP-Pは,P波として入射した波がフランジ底面で反射した後ち,再びP波として伝わったもの
であることを表す.同様に,P-Sは,P波からモード変換で転じたS波を意味する.このように,
それぞれの波面は,時間経過に伴う位置変化を追跡することで,対応する弾性波モードとモード
変換n履歴を記載することができる.
この図の詳細な説明は紙面の都合上本論文に譲り,ここでは,後の議論に関係する重要な
点だけを説明する.\\
図\ref{fig:fig3}(b)のRは,右方向に進展する表面波である.表面波は,入射点から当初
左右対称に同じスピードで伝わる.しかしなら,左側に進んだ表面波は,溶接ビードに到達した
時点で回折を起こし,余盛り表面を進む表面波と余盛り内部に浸入するS波に分離する.
これが,(a)においてR-Sと記した波面である.R-S波はそのまま継手内部を進み,本来き裂がある,
ルートギャップに向かって伝わる.R-S波は後でみるように非常に強い散乱波を励起する.
次に注目すべきモードは,(c)の時刻ではっせいしているP-P-Sの波面である.
これは,フランジ底面からの反射はP-Pがウェブに向かって進行する過程で,
余盛り表面で反射されることによって生じたS波である.
このことから,P-P-Sとラベル付している.P-P波は,余盛り表面での反射でP波,すなわち
P-P-Pも生じさえるが,特筆すべきはP-P-Sの挙動である.P-P-S波は,(d)の図では
余盛り表面から離れレンズ状の波面としてルートギャップ側に向かっている.
これは余盛りの形状を反映したもので,更に時間が経過した(e)では狭い範囲に集束して
大きな振幅を持つようになる.最後の(f)の時点ではこの傾向が一層すすみ,そのままルートギャップ近傍に
達する.これは,凸な余盛り形状が音響レンズのように作用して,横波を集束させたもので,
R-S波程ではないものの,き裂のあるモデルでは,強い散乱波を発生させる.
レンズ効果の強度は余盛り形状によって異なることは当然予想され,
もし凹な余盛りであれば逆のレンズ効果でこの波は拡散する.しかしながら,凸な余盛りであれば程度の差は
あるものの,レンズ効果が見込まれるため,その意味では普遍性のある現象と考えることができる.
\begin{figure}[h]
	\begin{center}
	\includegraphics[width=1.0\linewidth]{Figs/fig3.eps} 
	\end{center}
	\caption{
		入射波の進展挙動
	} 
	\label{fig:fig3}
\end{figure}
\subsection{き裂を有するモデルに対する結果(散乱波の挙動)}
次に,図\ref{fig:fig3}のような入射波に対して発生する散乱波の挙動を見る.
図\ref{fig:fig5},き裂を有するモデル(\ref{fig:fig2})に対する結果
から,入射波を差し引いて求めた散乱波成分だけの空間分布を示したものである.
スナップショットを示した6つの時刻は,図\ref{fig:fig3}と同じであるため,
入射波がどの位置に達した際,どのような散乱波が励起されるかが,
二つの図を対照させることで調べられる.例えば(a)で,P1とした波面は,
縦波散乱波で,図\ref{fig:fig3}(a)よりき裂の右側かP波が到来した直後
直後に発生したものであることが分かる.すなわち,送信点から直接
到来するP波に励起された,縦波後方散乱波ということが理解できる.
逆に,図\ref{fig:fig3}(d)では,溶接ビード内部を進んだR-S波が,
丁度,ウェブを通過して,本来き裂がある箇所に達しようとしている.
そこで,散乱波の分布を示す図\ref{fig:fig4}(d)と(e)を見ると,(d)では
存在しない横波S3が(e)の時刻に丁度発生している.これは,R-S波で励起された
散乱波であることは明らかで,この後,観測領域に向かって大きな振幅を持った
まま伝わることが確認できる.このように,入射波と散乱波の到達および発生
タイミングを見ることで,どのようなメカニズムで散乱波がいつ生じたかを
調べることができる.なお,前項で述べた,レンズ効果を受けて収束する
P-P-S波は,本稿では割愛するが,(f)の時刻直後にP波とS波波の散乱波両方を
右下方向に向けて発生させる.これが,フランジを伝わって超音波エコーとして
観測される.
%\begin{figure}[h]
%	\begin{center}
%	\includegraphics[width=1.0\linewidth]{Figs/fig4.eps} 
%	\end{center}
%	\caption{
%		き裂が存在するモデルでの解析結果
%	} 
%	\label{fig:fig4}
%\end{figure}
\begin{figure}[h]
	\begin{center}
	\includegraphics[width=1.0\linewidth]{Figs/fig5.eps} 
	\end{center}
	\caption{
		散乱波の発生と伝播状況
	} 
	\label{fig:fig5}
\end{figure}
\subsection{模擬探傷波形}
最後に,$0\leq x\leq 45 (y=12)$mmの観測範囲で得られた,フランジ表面の鉛直動分布を
図\ref{fig:fig6}に示す.この図にあるプロットは,横軸を時間,縦軸を位置(x)にとり,
各時空間点における鉛直速度を示した走時波形である.
これら3つの走時波形は上から順に(a)き裂のないモデルで計算した入射波成分,
(b)き裂を有するモデルで得られた全速度成分,および(c)二つの走時波形(b)と(c)の
差として得られた散乱波成分を示すものである.
超音波探傷ではこのうち(b)の走時波形が実際に計測でき,図\ref{fig:fig4}や図\ref{fig:fig5}
のような部材内部の状況を見ることはできない.また,入射波成分だけを実験で得る
ためには,き裂の無い正確なレプリカを作成し,正確に同じ位置で計測することが必要で,
実施は困難である.ここで,図\ref{fig:fig5}(c)の散乱波成分を見ると,
右上がりの直線が現れている.これは,R-S波で励起された散乱波で,
(b)の走時波形中でも容易に認識できる程度の強度がある.一方,18$\mu$付近で
はじめに現れる曲線は,余盛りのレンズ効果で集束された後散乱した横波であることが,
図\ref{fig:fig5}以後の時刻の散乱波の空間分布と照合することによって明らかにできる.
これ以外に,成因の特定できるものは,早い時間に到達す縦波後方散乱波P1がある.
P1は,他の波と時間的に重なることなく現れるという意味で有利だが,鉛直振動は他の散乱波
成分に比較して小さく,縦波の振動方向に合わせた高感度な受信センサーを用いる必要が
あることがここでの結果から理解できる.
\begin{figure}[h]
	\begin{center}
	\includegraphics[width=1.0\linewidth]{Figs/fig6.eps} 
	\end{center}
	\caption{
		探傷面上での模擬超音波波形(走時波形)
	} 
	\label{fig:fig6}
\end{figure}
%--------------------
\section{まとめ}
本研究では,T字溶接継手におけるき裂の超音波探傷試験を模擬した数値シミュレーションを
行い,き裂エコーの励起と伝播挙動を詳しく調べるともに,実際に観測しうるエコーを
模擬探傷波形において特定した.その結果,大きな振幅を持つ表面波や,早い時間帯に到達
する縦波散乱波が有望であることが分かった.さらに,余盛り表面のレンズ効果を受けて集束される
横波も散乱波の探傷波形中に比較的はっきりと現れることを示した.
なお,散乱波成分のみの探傷波形は実際には計測出来ないが,本研究で行ったように,
シミュレーションで予め発生位置と到達時刻を特定することがきれば,その測定に特価した
センサーの開発や送受条件の設定が可能となり,超音波探傷の高精度化や信頼性の向上に
有用な知見を提供することができる.今後は,実験と数値シミュレーションの比較を行い,
シミュレーションの有効性を検証することや,き裂向きや大きさを変化させたときの
探傷波形パターンの変化を調べることが課題となる.
%%%%%%%%%%%%%%%%%%%%%%%%%%%%%%%%%%%%%%%%%%%%%%%%%%%%%%%%%%%%%%%%%%%%%%%%%%%%%
%{\gt 謝辞:}
%本実験に用いた花崗岩供試体は浮田石材店代表浮田隆司氏に提供頂いた.
%また本研究の推進には,科学研究費補助金(基盤研究©課題番号\#18K04334)の補助を受けた.
%併せて謝意を表す.
%%%%%%%%%%%%%%%%%%%%%%%%%%%%%%%%%%%%%%%%%%%%%%%%%%%%%%%%%%%%%%%%%%%%%%%%%%%%%
\end{document}
%\newpage
%\lastpagecontrol[2cm]{13.7cm}
\begin{thebibliography}{99}
\begin{spacing}{1.175}
\bibitem{Itagaki}
	板垣 昌幸:電気化学インピーダンス法 第二版 原理・測定・解析,丸善出版, 2011. 
\end{spacing}
\end{thebibliography}
%\begin{flushright}
%	\small
%	\bf{ (Received June 24, 2020)\\
%	(Accepted November 19, 2020)}
%\end{flushright}
\end{document}

%\lastpagesettings
%\begin{minipage}[c]{13.7cm}
%\end{minipage}
%\lastpagecontrol[0cm]{13.7cm}
%\begin{multicols}{1}
%-------------------------------------------------
%-------------------------------------------------
%\end{multicols}

